\documentclass[10pt,a4paper]{article}
\usepackage[latin1]{inputenc}
\usepackage{amsmath}
\usepackage{amsfonts}
\usepackage{amssymb}
\usepackage{commands}
\usepackage[inference]{semantic}
\usepackage{listings}

% uncomment next line to restore colors
% \def\withcolor{}

\ifdefined\withcolor
	\definecolor{haskellblue}{rgb}{0.0, 0.0, 1.0}
	\definecolor{haskellblue}{rgb}{1.0, 0.0, 0.0}
	\definecolor{gray_ulisses}{gray}{0.55}
	\definecolor{castanho_ulisses}{rgb}{0.71,0.33,0.14}
	\definecolor{preto_ulisses}{rgb}{0.41,0.20,0.04}
	\definecolor{green_ulisses}{rgb}{0.0,0.4,0.0}
\else
	\definecolor{haskellblue}{gray}{0.1}
	\definecolor{haskellred}{gray}{0.1}
	\definecolor{gray_ulisses}{gray}{0.1}
	\definecolor{castanho_ulisses}{gray}{0.1}
	\definecolor{preto_ulisses}{gray}{0.1}
	\definecolor{green_ulisses}{gray}{0.1}
\fi


\def\codesize{\normalsize}

\lstdefinelanguage{HaskellUlisses} {
	basicstyle=\ttfamily\codesize,
	sensitive=true,
	morecomment=[l][\color{gray_ulisses}\ttfamily\codesize]{--},
	%% morecomment=[s][\color{gray_ulisses}\ttfamily\codesize]{\{-}{-\}},
	morestring=[b]",
	stringstyle=\color{haskellred},
	showstringspaces=false,
	numberstyle=\codesize,
	numberblanklines=true,
	showspaces=false,
	breaklines=true,
	showtabs=false,
	emph=
	{[1]
		FilePath,IOError,abs,acos,acosh,all,and,any,appendFile,approxRational,asTypeOf,asin,
		asinh,atan,atan2,atanh,basicIORun,break,catch,ceiling,chr,compare,concat,concatMap,
		const,cos,cosh,curry,cycle,decodeFloat,denominator,digitToInt,div,divMod,drop,
		dropWhile,either,elem,encodeFloat,enumFrom,enumFromThen,enumFromThenTo,enumFromTo,
		error,even,exp,exponent,fail,filter,flip,floatDigits,floatRadix,floatRange,floor,
		fmap,foldl,foldl1,foldr,foldr1,fromDouble,fromEnum,fromInt,fromInteger,
		fromRational,fst,gcd,getChar,getContents,getLine,head,id,inRange,index,init,intToDigit,
		interact,ioError,isAlpha,isAlphaNum,isAscii,isControl,isDenormalized,isDigit,isHexDigit,
		isIEEE,isInfinite,isLower,isNaN,isNegativeZero,isOctDigit,isPrint,isSpace,isUpper,iterate,
		last,lcm,length,lex,lexDigits,lexLitChar,lines,log,logBase,lookup,map,mapM,mapM_,max,
		maxBound,maximum,maybe,min,minBound,minimum,mod,negate,not,notElem,numerator,odd,
		or,pi,pred,primExitWith,print,product,properFraction,putChar,putStr,putStrLn,quot,
		quotRem,range,rangeSize,read,readDec,readFile,readFloat,readHex,readIO,readInt,readList,readLitChar,
		readLn,readOct,readParen,readSigned,reads,readsPrec,realToFrac,recip,rem,repeat,replicate,
		reverse,round,scaleFloat,scanl,scanl1,scanr,scanr1,seq,sequence,sequence_,show,showChar,showInt,
		showList,showLitChar,showParen,showSigned,showString,shows,showsPrec,significand,signum,sin,
		sinh,snd,span,splitAt,sqrt,subtract,succ,sum,tail,take,takeWhile,tan,tanh,threadToIOResult,toEnum,
		toInt,toInteger,toLower,toRational,toUpper,truncate,uncurry,undefined,unlines,until,unwords,unzip,
		unzip3,userError,words,writeFile,zip,zip3,zipWith,zipWith3,listArray,doParse,for,initTo,
        maxEvens,create,get,set,initialize,idVec,fastFib,fibMemo,
        insert,union,split,size,fromList,initUpto,trim,quickSort,insertSort,append,upperCase,
        copy, group, doDownLoop, mapAccumR, peekByteOff,
        pokeByteOff,spanByte, 
        good, bad, foo, explode, 
        fib, ack, 
        tLen,
        memcpy,writeChar,unsafeWrite,unsafeFreeze,
        singleton
	},
	emphstyle={[1]\color{haskellblue}},
	emph=
	{[2]
		Bool,Char,Double,Either,Float,IO,Integer,Int,Maybe,Ordering,Rational,Ratio,ReadS,ShowS,String,
		Word8,Nat,NonZero,Nat64,Text,ByteString,ByteStringSZ,ByteStringN,
        Ptr,ForeignPtr,CSize
        InPacket,Tree,Prop,TreeEq,TreeLt,Vec,
        NullTerm,IncrList,DecrList,UniqList,BST,MinHeap,MaxHeap,
        PtrN,ByteStringN,ByteStringEq,VO,ByteStringsEq,ByteStringNE
	},
	emphstyle={[2]\color{castanho_ulisses}},
	emph=
	{[3]
		case,class,data,deriving,do,else,if,import,in,infixl,infixr,instance,let,
		module,measure,predicate,of,primitive,then,refinement,type,where
	},
	emphstyle={[3]\color{preto_ulisses}\textbf},
	emph=
	{[4]
		quot,rem,div,mod,elem,notElem,seq
	},
	emphstyle={[4]\color{castanho_ulisses}\textbf},
	emph=
	{[5]
		PS,Tip,Node,EQ,False,GT,Just,LT,Left,Nothing,Right,True,Show,Eq,Ord,Num
	},
	emphstyle={[5]\color{green_ulisses}}
}

%%%ORIG
%%%\lstnewenvironment{code}
%%%{\textbf{Haskell Code} \hspace{1cm} \hrulefill \lstset{language=HaskellUlisses}}
%%%{\hrule\smallskip}

%V1
%\lstnewenvironment{code}
%{\smallskip \lstset{language=HaskellUlisses}}
%{\smallskip}

\lstnewenvironment{code}
{\lstset{language=HaskellUlisses}}
{}

\lstMakeShortInline[language=HaskellUlisses]@



\newcommand\hastype[3]{\ensuremath{#1 \vdash #2 \colon #3}}
\newcommand\iswellformed[2]{\ensuremath{#1 \vdash #2}}
\newcommand\biswellformed[2]{\ensuremath{#1 \vdash_B #2}}
\newcommand\issubtype[3]{\ensuremath{#1 \vdash #2 \preceq #3}}
\newcommand\issubref[3]{\ensuremath{#1 \vdash #2 \Rightarrow #3}}
\title{Towards the most general type}
\begin{document}
%\maketitle
\section*{Liquid Type Inference}
Liquid type inference has two important features, 
namely logical qualifiers and refinement variables.

\subsection*{Refinement Variables}
Liquid Types use the common rules (see FORMALISM) for type-checking and subtyping 
but whenever the system should \textbf{guess} a type
they use haskell's type as template to create a liquid type with 
variables as refinements.

As an example, in the rule
$$
\inference{
	\hastype{\Gamma, x\colon\tau_x}{e}{\tau} &&
	\iswellformed{\Gamma}{\tau_x}
}{
	\hastype{\Gamma}{\lambda x . e}{(x:\tau_x \rightarrow \tau)}
}
$$

Say that haskell's (unrefined) type of the variable $x$
is a type variable $a$, 
then the liquid type will be 
$\left\lbrace v\colon a \mid k_x \right\rbrace$
and the well formedness constraint
$\iswellformed{\Gamma}{\left\lbrace v\colon a \mid k_x \right\rbrace}$
will state that $k_x$ can \textbf{depend} on any variable that exists on $\Gamma$.


\subsection*{Qualifiers}
But what can $k_x$ be solved to?
It should be solved to a logical predicate.
Since we can not search every possible predicate, 
it will be a conjunction of \textit{logical qualifiers}.
%
Logical qualifiers,  
are given as input to the solver and are of the following form
%
$$
\begin{array}{rrcl}
\emphbf{Qualifiers} \quad 
  & q
  & ::= 
  &    x(\overline{x}) := p   
  \\[0.05in] 

\emphbf{Predicates} \quad 
  & p 
  & ::= 
  &      e_r [> | \geq | < | \leq | == | !=] e_r
  \spmid  p [\lor | \land | \Rightarrow | \Leftrightarrow ] p 
  \spmid \lnot p
  \\[0.05in] 

\emphbf{Expressions} \quad 
  & e_r
  & ::= 
  &    x \spmid c  \spmid  e_r [+ | -] e_r   
\end{array}
$$
%
As an example, the logical qualifier 
$q(x) = v > x$
compares the qualified value $v$ with the variable $x$.

\subsection*{Well-formedness Constraints}
We create and split well-formedness rules, as usual.
%
Using the basic well-formedness constraints \texttt{WFC} for all refinement variables 
and the set of qualifiers \texttt{Q} we decide 
all the possible candidates of the variables as follows:
$$
cand(k) = \left\lbrace 
q(\overline{x}) 
	\mid \biswellformed{\Gamma}{\left\lbrace v\colon\tau \mid k \right\rbrace}\in \texttt{WFC}
	\land \hastype{\Gamma, v:\tau}{q(\overline{x})}{bool}
\right\rbrace
$$

We note that in $q(\overline{x})$ the variables  
$\overline{x}$ instantiate the qualifier's variables with 
concrete ones that appear in the well-formedness constraints.
Also,
typechecking \hastype{\Gamma}{e}{\tau}
is the usual unrefined typechecking rule.

$$
\begin{array}{rrcl}
\emphbf{Well-Formedness Constraints} \quad 
  & \texttt{WFC}
  & ::= 
  &     \emptyset \spmid \biswellformed{\Gamma}{\tau},\texttt{WFC}    
  \\[0.05in] 

\emphbf{Qualifiers} \quad 
  & \texttt{Q}
  & ::= 
  &      \emptyset \spmid q(\overline{x}), \texttt{Q}
\end{array}
$$


\subsection*{Sub-Typing Constraints}
We create and split well-formedness rules, as usual (see FORMALISM).
%
As an example, in the rule
$$
\inference{
	\hastype{\Gamma}{e_1}{(x:\tau_{x1} \rightarrow \tau)} &&
	\hastype{\Gamma}{e_2}{\tau_{x2}} &&
	\issubtype{\Gamma}{\tau_{x2}}{\tau_{x1}}
}{
	\hastype{\Gamma}{e_1 \ e_2}{\tau}
}
$$
We state that the type of the argument $\tau_{x2}$ should be a substype
of the argument of the function $\tau_{x1}$.
% 
Through substyping rules, constraints split down to implication checking of the form
%
$\issubref{\Gamma}{r_1}{r_2}$
%
Where $r$ is a refinement, ie., a list of either a refinement variable or a predicate.

$$
\begin{array}{rrcl}
\emphbf{Implication Constraints} \quad 
  & \texttt{SUB}
  & ::= 
  &     \emptyset \spmid \issubref{\Gamma}{r}{r},\texttt{SUB}    
  \\[0.05in] 

\emphbf{Refinement} \quad 
  & \texttt{r}
  & ::= 
  &     \emptyset \spmid q, r    
  \\[0.05in] 

\emphbf{Basic Refinement} \quad 
  & \texttt{q}
  & ::= 
  &      k \spmid p
\end{array}
$$

\subsection*{Solving Implications}
We use the below fixpoint algorithm to solve refinement variables:

\begin{code}
st := {k, cand(k)};

update:
 forall G |- r1 => r2 in SUB
  let p1 := {p | p in r1} `union` {st(k) | k in r1}
  forall k in r2 
   st(k) := st(k) `intersection` p1
 if st is updated go to update	    
\end{code}

Finally, we use a solver to prove that all constraints are satisfied.
%
%
%In this algorithm the refinement variables start from the stronger solution 
%and at each iteration they can get weaker so at to satisfy the constraints.
%%
%So, the final solution we get is the stronger with respect to the constraints.


\section*{FORMALISM}

$$
\begin{array}{rrcl}
\emphbf{Expressions} \quad 
  & e
  & ::= 
  &     x \spmid c \spmid \lambda x . e \spmid e \ e     
  \\[0.05in] 

\emphbf{Refined Types} \quad 
  & \tau
  & ::=  
  & 	\left\lbrace v : b \mid r \right\rbrace
         \spmid x:\tau \rightarrow \tau    
  \\[0.05in] 

\emphbf{Environment} \quad 
  & \Gamma
  & ::= 
  &      \emptyset \spmid x\colon \tau, \Gamma
\end{array}
$$

\hfill\mbox{\hastype{\Gamma}{e}{\tau}}
$$
\inference{
	(x, \left\lbrace v : b \mid r \right\rbrace) \in \Gamma
}{
	\hastype{\Gamma}{x}{\left\lbrace v : b \mid v = x \right\rbrace}
}
\qquad
\inference{
	(x, y:\tau_y\rightarrow\tau) \in \Gamma
}{
	\hastype{\Gamma}{x}{(y\colon\tau_y\rightarrow\tau)}
}
\qquad
\inference{
	\hastype{\Gamma, x\colon\tau_x}{e}{\tau} &&
	\iswellformed{\Gamma}{\tau_x}
}{
	\hastype{\Gamma}{\lambda x . e}{(x:\tau_x \rightarrow \tau)}
}
$$
$$
\inference{
}{
	\hastype{\Gamma}{c}{ty(c)}
}
\qquad
\inference{
	\hastype{\Gamma}{e_1}{(x:\tau_{x1} \rightarrow \tau)} &&
	\hastype{\Gamma}{e_2}{\tau_{x2}} &&
	\issubtype{\Gamma}{\tau_{x2}}{\tau_{x1}}
}{
	\hastype{\Gamma}{e_1 \ e_2}{\tau}
}
$$

\hfill\mbox{\iswellformed{\Gamma}{\tau}}
$$
\inference{
	\biswellformed{\Gamma}{\left\lbrace v : b \mid r \right\rbrace}
}{
	\iswellformed{\Gamma}{\left\lbrace v : b \mid r \right\rbrace}
}
\qquad
\inference{
	\iswellformed{\Gamma}{\tau_x} &&
	\iswellformed{\Gamma, x \colon \tau_x}{\tau}
}{
	\iswellformed{\Gamma}{x:\tau_x \rightarrow \tau}
}
$$

\hfill\mbox{\issubtype{\Gamma}{\tau}{\tau}}
$$
\inference{
	\issubref{\Gamma, v:b}{r}{r'}
}{
	\issubtype{\Gamma}{\left\lbrace v : b \mid r \right\rbrace}{\left\lbrace v : b \mid r' \right\rbrace}
}
\qquad
\inference{
	\issubtype{\Gamma}{\tau'_x}{\tau_x} &&
	\issubtype{\Gamma, x \colon \tau'_x}{\tau}{\tau'}
}{
	\issubtype{\Gamma}{x:\tau_x \rightarrow \tau}{x:\tau'_x \rightarrow \tau'}
}
$$

\end{document}