	%%%%%%%%%%%%%%%%%%%%%%%%%%%%%%%%%%%%%%%%%%%%%%%%%%%%%%%%%%%%%%%%%%%%%%%%
%%%%%%%%%%%%%%%%%%%%%% Simple LaTeX CV Template %%%%%%%%%%%%%%%%%%%%%%%%
%%%%%%%%%%%%%%%%%%%%%%%%%%%%%%%%%%%%%%%%%%%%%%%%%%%%%%%%%%%%%%%%%%%%%%%%

%%%%%%%%%%%%%%%%%%%%%%%%%%%%%%%%%%%%%%%%%%%%%%%%%%%%%%%%%%%%%%%%%%%%%%%%
%% NOTE: If you find that it says                                     %%
%%                                                                    %%
%%                           1 of ??                                  %%
%%                                                                    %%
%% at the bottom of your first page, this means that the AUX file     %%
%% was not available when you ran LaTeX on this source. Simply RERUN  %%
%% LaTeX to get the ``??'' replaced with the number of the last page  %%
%% of the document. The AUX file will be generated on the first run   %%
%% of LaTeX and used on the second run to fill in all of the          %%
%% references.                                                        %%
%%%%%%%%%%%%%%%%%%%%%%%%%%%%%%%%%%%%%%%%%%%%%%%%%%%%%%%%%%%%%%%%%%%%%%%%

%%%%%%%%%%%%%%%%%%%%%%%%%%%% Document Setup %%%%%%%%%%%%%%%%%%%%%%%%%%%%

% Don't like 10pt? Try 11pt or 12pt
\documentclass[11pt]{article}

% This is a helpful package that puts math inside length specifications
\usepackage{calc}

% Simpler bibsection for CV sections
% (thanks to natbib for inspiration)
\makeatletter
\newlength{\bibhang}
\setlength{\bibhang}{1em}
\newlength{\bibsep}
 {\@listi \global\bibsep\itemsep \global\advance\bibsep by\parsep}
\newenvironment{bibsection}
    {\minipage[t]{\linewidth}\list{}{%
        \setlength{\leftmargin}{\bibhang}%
        \setlength{\itemindent}{-\leftmargin}%
        \setlength{\itemsep}{\bibsep}%
        \setlength{\parsep}{\z@}%
        }}
    {\endlist\endminipage}
\makeatother

% Layout: Puts the section titles on left side of page
\reversemarginpar

%
%         PAPER SIZE, PAGE NUMBER, AND DOCUMENT LAYOUT NOTES:
%
% The next \usepackage line changes the layout for CV style section
% headings as marginal notes. It also sets up the paper size as either
% letter or A4. By default, letter was used. If A4 paper is desired,
% comment out the letterpaper lines and uncomment the a4paper lines.
%
% As you can see, the margin widths and section title widths can be
% easily adjusted.
%
% ALSO: Notice that the includefoot option can be commented OUT in order
% to put the PAGE NUMBER *IN* the bottom margin. This will make the
% effective text area larger.
%
% IF YOU WISH TO REMOVE THE ``of LASTPAGE'' next to each page number,
% see the note about the +LP and -LP lines below. Comment out the +LP
% and uncomment the -LP.
%
% IF YOU WISH TO REMOVE PAGE NUMBERS, be sure that the includefoot line
% is uncommented and ALSO uncomment the \pagestyle{empty} a few lines
% below.
%

%% Use these lines for letter-sized paper
\usepackage[paper=letterpaper,
            %includefoot, % Uncomment to put page number above margin
            marginparwidth=1.2in,     % Length of section titles
            marginparsep=.05in,       % Space between titles and text
            margin=1in,               % 1 inch margins
            includemp]{geometry}

%% Use these lines for A4-sized paper
%\usepackage[paper=a4paper,
%            %includefoot, % Uncomment to put page number above margin
%            marginparwidth=30.5mm,    % Length of section titles
%            marginparsep=1.5mm,       % Space between titles and text
%            margin=25mm,              % 25mm margins
%            includemp]{geometry}

%% More layout: Get rid of indenting throughout entire document
\setlength{\parindent}{0in}

%% This gives us fun enumeration environments. compactitem will be nice.
\usepackage{paralist}

%% Reference the last page in the page number
%
% NOTE: comment the +LP line and uncomment the -LP line to have page
%       numbers without the ``of ##'' last page reference)
%
% NOTE: uncomment the \pagestyle{empty} line to get rid of all page
%       numbers (make sure includefoot is commented out above)
%
\usepackage{fancyhdr,lastpage}
\pagestyle{fancy}
%\pagestyle{empty}      % Uncomment this to get rid of page numbers
\fancyhf{}\renewcommand{\headrulewidth}{0pt}
\fancyfootoffset{\marginparsep+\marginparwidth}
\newlength{\footpageshift}
\setlength{\footpageshift}
          {0.5\textwidth+0.5\marginparsep+0.5\marginparwidth-2in}
\lfoot{\hspace{\footpageshift}%
       \parbox{4in}{\, \hfill %
                    \arabic{page} of \protect\pageref*{LastPage} % +LP
%                    \arabic{page}                               % -LP
                    \hfill \,}}

% Finally, give us PDF bookmarks
\usepackage{color,hyperref}
\definecolor{darkblue}{rgb}{0.0,0.0,0.3}
\hypersetup{colorlinks,breaklinks,
            linkcolor=darkblue,urlcolor=darkblue,
            anchorcolor=darkblue,citecolor=darkblue}

%%%%%%%%%%%%%%%%%%%%%%%% End Document Setup %%%%%%%%%%%%%%%%%%%%%%%%%%%%


%%%%%%%%%%%%%%%%%%%%%%%%%%% Helper Commands %%%%%%%%%%%%%%%%%%%%%%%%%%%%

% The title (name) with a horizontal rule under it
%
% Usage: \makeheading{name}
%
% Place at top of document. It should be the first thing.
\newcommand{\makeheading}[1]%
        {\hspace*{-\marginparsep minus \marginparwidth}%
         \begin{minipage}[t]{\textwidth+\marginparwidth+\marginparsep}%
                {\large \bfseries #1}\\[-0.15\baselineskip]%
                 \rule{\columnwidth}{1pt}%
         \end{minipage}}

% The section headings
%
% Usage: \section{section name}
%
% Follow this section IMMEDIATELY with the first line of the section
% text. Do not put whitespace in between. That is, do this:
%
%       \section{My Information}
%       Here is my information.
%
% and NOT this:
%
%       \section{My Information}
%
%       Here is my information.
%
% Otherwise the top of the section header will not line up with the top
% of the section. Of course, using a single comment character (%) on
% empty lines allows for the function of the first example with the
% readability of the second example.
\renewcommand{\section}[2]%
        {\pagebreak[2]\vspace{1.3\baselineskip}%
         \phantomsection\addcontentsline{toc}{section}{#1}%
         \hspace{0in}%
         \marginpar{
         \raggedright \scshape #1}#2}

% An itemize-style list with lots of space between items
\newenvironment{outerlist}[1][\enskip\textbullet]%
        {\begin{itemize}[#1]}{\end{itemize}%
         \vspace{-.6\baselineskip}}

% An environment IDENTICAL to outerlist that has better pre-list spacing
% when used as the first thing in a \section
\newenvironment{lonelist}[1][\enskip\textbullet]%
        {\vspace{-\baselineskip}\begin{list}{#1}{%
        \setlength{\partopsep}{0pt}%
        \setlength{\topsep}{0pt}}}
        {\end{list}\vspace{-.6\baselineskip}}

% An itemize-style list with little space between items
\newenvironment{innerlist}[1][\enskip\textbullet]%
        {\begin{compactitem}[#1]}{\end{compactitem}}

% An environment IDENTICAL to innerlist that has better pre-list spacing
% when used as the first thing in a \section
\newenvironment{loneinnerlist}[1][\enskip\textbullet]%
        {\vspace{-\baselineskip}\begin{compactitem}[#1]}
        {\end{compactitem}\vspace{-.6\baselineskip}}

% To add some paragraph space between lines.
% This also tells LaTeX to preferably break a page on one of these gaps
% if there is a needed pagebreak nearby.
\newcommand{\blankline}{\quad\pagebreak[2]}

% Uses hyperref to link DOI
\newcommand\doilink[1]{\href{http://dx.doi.org/#1}{#1}}
\newcommand\doi[1]{doi:\doilink{#1}}


%%%%%%%%%%%%%%%%%%%%%%%% End Helper Commands %%%%%%%%%%%%%%%%%%%%%%%%%%%

%%%%%%%%%%%%%%%%%%%%%%%%% Begin CV Document %%%%%%%%%%%%%%%%%%%%%%%%%%%%

\begin{document}
\makeheading{Niki Vazou}

\section{Personal Information}
%
% NOTE: Mind where the & separators and \\ breaks are in the following
%       table.
%
% ALSO: \rcollength is the width of the right column of the table
%       (adjust it to your liking; default is 1.85in).
%
\newlength{\rcollength}\setlength{\rcollength}{0.85in}%
%
\begin{tabular}[t]{ll}
\textit{Born Date:}& July 20th, 1987\\
\textit{Citizenship:} \hspace{5cm} & Greece\\
\textit{Cell Phone:} &(858)405-4399\\
\textit{E-mail:} &\href{mailto:nvazou@cs.ucsd.edu}{nvazou@cs.ucsd.edu}\\
\textit{Website:} & \href{http://goto.ucsd.edu/~nvazou}{goto.ucsd.edu/~nvazou}
\end{tabular}


\section{Research Interests}
%
Programming Languages Design and Implementation,\\
Functional Programming, Program Verification, Type Systems,\\
L-calculus, Mathematical logic and Formal Languages

\section{Education}
%\begin{list}{\labelitemi}{\leftmargin=0em}{\labelsep=0em}
PhD (September 2011-present)
%\item[]M.S.

\begin{innerlist}
\item[]
	\href{http://www.ucsd.edu/}{\textbf{University of California, San Diego}},\\
    \href{http://www.ece.ucsd.edu/}
             {Electrical and Computer Engineering}
        \begin{innerlist}
        \item GPA: 3.962/4
        \item Current Project: \emph{Liquid Types in Haskell}
        \item Supervisor:
              \href{http://goto.ucsd.edu/~rjhala/}
                   {Ranjit Jhala}
        \item Research Area: Programming Languages
        \end{innerlist}
\end{innerlist}

\blankline

Diploma (November 2010)
\begin{innerlist}
\item[]
	\href{http://www.ntua.gr/}{\textbf{National Technical University of Athens}},\\
    \href{http://www.ece.ntua.gr/}
             {Electrical and Computer Engineering}
        \begin{innerlist}
        \item GPA: 9.24/10
        \item Majors: Computer Software and Computer Systems
        \item Minors: Mathematics and Signal, Automatic Control and Robotics
        \item Thesis Topic: \emph{Type Systems with Linear Capabilities}
        \item Supervisor:
              \href{http://www.softlab.ntua.gr/~nickie/}
                   {Nikolaos S. Papaspyrou}
        \item Research Area: Programming Languages
        \end{innerlist}
\end{innerlist}

\blankline

High school (July 2005)
\begin{innerlist}
\item[]$3^{nd}$ High School of Korydallos
\item[]
        \begin{innerlist}
        \item GPA: 19.32/20
        \item Rank: First in School
        \end{innerlist}

\end{innerlist}
%\end{list}

\section{Publications}
%
\begin{innerlist}
\item[] Niki Vazou, Eric Seidel, and Ranjit Jhala,
\emph{Demo: Liquid Types for Haskell}, Haskell 2013

\blankline

\item[] Niki Vazou, Patric M. Rondon, Eric Seidel, and Ranjit Jhala,
\emph{Tutorial: Type-Based Analysis of Higher-Order Programs}, HOPA 2013

\blankline

\item[] Niki Vazou, Patric M. Rondon, and Ranjit Jhala,
\emph{Abstract Refinements Types}, ESOP 2013

\blankline

\item[] Niki Vazou, Michalis A. Papakyriakou, and Nikolaos Papaspyrou, 
\emph{Memory Safety and Race Freedom in Concurrent Programming Languages 
with Linear Capabilities}, FedCSIS 2011
\end{innerlist}

\blankline

\section{Teaching Assistant}
\begin{innerlist}
\item[]University of California, San Diego
	\begin{innerlist}
	\item \href{http://cseweb.ucsd.edu/classes/wi13/cse230-a/}
	           {Programming Languages} 
	          \hfill \textbf{Winter 2013}
	\end{innerlist}
\end{innerlist}
\blankline
\begin{innerlist}
\item[]National Technical University of Athens
	\begin{innerlist}
	\item \href{http://courses.softlab.ntua.gr/progintro/}
	           {Computer Programming} 
	          \hfill \textbf{Fall 2010}
	\end{innerlist}
\end{innerlist}


\section{Professional Experience}
\begin{innerlist}
\item[]\href{http://research.microsoft.com/en-us/labs/cambridge/}{Microsoft Research}, Cambridge, UK \hfill \textbf{September 2013 - Present}\\

\blankline

\item[]\href{http://opalang.org/}{Opa}, Paris, France \hfill \textbf{June - September 2012}\\
I refined the error reporting of Opalang's type system and 
added some features in Opa's blog. 
\end{innerlist}

\section{Languages} 
\begin{innerlist}
\item[] Greek (native speaker)
\item[] English (\href {http://www.cambridgeesol.gr/exams/general-english/cpe.html}{CPE})
\item[] Italian (moderate)
\end{innerlist}

\end{document}

\section{References}
\begin{innerlist}
\item[]\href{http://goto.ucsd.edu/~rjhala/}{Ranjit Jhala}
 	\begin{innerlist}
 	\item Prof. Department of Computer Science and Engineering
    \item University of California, San Diego
 	\item Phone: (858) 534 1420
	\item E-mail: jhala AT cs DOT ucsd DOT edu
 	\end{innerlist}

\item[]\href{http://www.softlab.ntua.gr/~nickie/}{Nikolaos S. Papaspyrou}
 	\begin{innerlist}
 	\item Prof. School of Electrical and Computer Engineering
	\item National Technical University of Athens
	\item Phone: +30-210-7723393
	\item E-mail: nickie AT softlab DOT ntua DOT gr
 	\end{innerlist}
\end{innerlist}

%%%%%%%%%%%%%%%%%%%%%%%%%% End CV Document %%%%%%%%%%%%%%%%%%%%%%%%%%%%%
