\section*{Proposed Work}

The goal of this project is to create a sound and usable verifier
for real-world haskell applications that can be embedded to an
existing haskell compiler.
%
Many concrete subgoals exist towards this direction: \\

\textbf{Soundness}
\begin{itemize}
\item
As discussed earlier liquidHaskell's soundeness
depends on a termination-checker and
describing infinite structures is impossible.
%
Since infinite structures appear often in 
idiomatic haskell programming
we would like to cut this dependency, 
without sacrificing any of the current 
features (ie., precision, expressiveness) of the tool.


%  A direct approach is to alter the meaning of a refinement type e:{v:a | p}
%  to if computation of e terminates, then the result satisfies p.
%  Though sound, this information is too general to lead to precise results.
%  But, using the termination checker we can have types 
%  `e:{v:a | | p}`
%  stating that the computation of e does terminate and the result satisfy p

\item 
Defining the formal semantics of liquidHaskell
is a long term goal, 
as it will provide both to us and the users 
a better understanding of the tool
\end{itemize}


\textbf{Usability}
\begin{itemize}
\item Currently, typechecking is performed in granularity of module (haskell file)
  as the type of each function depends on its call sites.
  We want to cut this dependency and get both modularity and speedup.
  Moreover, 
  typechecking will become parallelizable and incremental, ie., 
  local modifications to the source code, will require rechecking
  only a small amount of code.

%\item Multiple types 
%  In liquidHaskell, currently, each expression has a unique type.
%  Many times, we want to use different types on different occurrences of the same expression.
%  thus support of conjunction types seems inevitable.

\item 
  Some standard requirements appear often when verifying code. 
  For example, when defining a new data type we need to describe 
  its size,
  or many times, to enable verification we need to inline higher order functions.
  LiquidHaskell can automatically perform these standard procedures.
  
%\item Abstract refinement inference
%  As noted earlier liquid haskell supports abstract refinements to parameterize function types.
%  It would be useful if abstract refinement types can be infered. 
%  It seems though, that blindly trying to parameterize each type will be too expensive, 
%  thus some heuristics should be used.


%\item Tutorial:
%  Currently the tool can be found online combined with a number of examples, 
%  and blog posts on how to use it.
%  Since this is a work in progress many of its features are added or modified.
%  It would be nice to construct and maintain a up-to-date document 
%  that combines both how to use liquidHaskell and our experience so far in using the tool. 

\item Error reporting and diagnosis are crucial in  
Currently, when liquidHaskell fails to prove some specification, 
it reports the location of the code where typechecking fails .
%
  It turns out that then the user needs to perform some reasoning to understand the source of this failure.
  As noted in Abductive Inference [PLDI 12], most of this reasoning is already performed 
  by the tool. 
  %
  It would be ideal to expose some of these information 
  either by providing informing error messages or 
  by interactively helping the user identify the source of the error.
\end{itemize}

\textbf{Real-world Applications}
\begin{itemize}

\item Liquidhaskell has been used to verify code from the wild.
As a next step we can use it to verify hot applications
like security properties in web applications
or data-race freedom in concurrent ones.

\item 
  We would like to investigate how liquidHaskell performs when verifying haskell idiomatic 
  code that depends on features like type classes, monads or infinite structures.
  For instance, we would like to verify and use invariants of the state 
  in monadic functions.
\end{itemize}

\textbf{Integration to a real compiler.}
\begin{itemize}
\item LiquidHaskell can run at compile time to track common 
 errors. 
For instance, it can be used to identify partially defined function, 
 that can lead to run-time exceptions or 
 that user code satisfies library function invariants, ie,.
 it never takes a the head of an empty list.
 
\item LiquidHaskell can interact with haskell's dependent types.
  While being very expressive, dependent types require the user to 
  explicitly provide proofs.
  A nice research direction is to investigate 
  whether liquidHaskell can be used to automatically infer
  some of the required proofs.
  \end{itemize}

In a nutshell, the ultimate goal is to create a 
user-friendly, efficient, and sound tool 
so as to integrate formal verification, via liquidHaskell, 
into the development chain of haskell applications.

