Assume \hastype{\emptyset}{e}{\tau}.
We will prove the Lemma by induction on the derivation tree.
\begin{itemize}
\item Case \rtexact. Assume
$$	\hastype{\emptyset}{e}{\tref{v}{b}{v = e}}
$$
where $\tau\equiv\tref{v}{b}{v=e}$.
By inversion
$$	\hastype{\emptyset}{e}{\tref{v}{b}{e'}}$$
By IH 
either $e \equiv v$ or there exists an $e'$ such that \eval{e}{e'}.

\item Cases \rtvarbase, \rtvar cannot occur, as $\Gamma = \emptyset$
\item Cases \rtconst and \rtfun are trivial, 
		as $e$ is a value
\item Case \rtsub. Assume 
$$\hastype{\emptyset}{e}{\tau}$$
By inversion
$$	\hastype{\emptyset}{e}{\tau'}$$
By IH 
either $e \equiv v$ or there exists an $e'$ such that \eval{e}{e'}.
\item Case \rtapp. Assume 
$$\hastype{\emptyset}{e}{\tau}\ (1)$$
where $e\equiv\eapp{e_1}{e_2}$ and $\tau\equiv\tau'\sub{x}{e_2}$
By inversion
$$
	\hastype{\emptyset}{e_1}{(\tfun{x}{\tau_{x}}{\tau})}\ (2)\qquad
	\hastype{\emptyset}{e_2}{\tau_{x}}\ (3)
$$

We split cases on the structure of $e$.
\begin{itemize}
\item $e\equiv \eapp{c}{v_2}$.
Then, $e'\equiv\interp{c}(v_2)$.

\item $e\equiv \eapp{c}{e_2}$ where $e_2$ is not a value, 
By IH on $(3)$ \eval{e_2}{e_2'} and  $e' \equiv \eapp{e_1}{e_2'}$

\item $e \equiv \eapp{\efun{x}{e_x}}{e_2}$.
Then, $e' \equiv e_x\sub{x}{e_2}$.

\item $e \equiv \eapp{e_1}{e_2}$, where $e_1$ is not a value.
Then, by IH on $(2)$ \eval{e_1}{e_1'} and 
$e'\equiv\eapp{e_1'}{e_2}$.
\end{itemize}
\end{itemize}