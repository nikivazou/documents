\documentclass[10pt,a4paper]{article}
\usepackage[latin1]{inputenc}
\usepackage{amsmath}
\usepackage{amsfonts}
\usepackage{amssymb}
\usepackage{commands}
\usepackage{xspace}
\usepackage[inference]{semantic}
\usepackage{listings}

% uncomment next line to restore colors
% \def\withcolor{}

\ifdefined\withcolor
	\definecolor{haskellblue}{rgb}{0.0, 0.0, 1.0}
	\definecolor{haskellblue}{rgb}{1.0, 0.0, 0.0}
	\definecolor{gray_ulisses}{gray}{0.55}
	\definecolor{castanho_ulisses}{rgb}{0.71,0.33,0.14}
	\definecolor{preto_ulisses}{rgb}{0.41,0.20,0.04}
	\definecolor{green_ulisses}{rgb}{0.0,0.4,0.0}
\else
	\definecolor{haskellblue}{gray}{0.1}
	\definecolor{haskellred}{gray}{0.1}
	\definecolor{gray_ulisses}{gray}{0.1}
	\definecolor{castanho_ulisses}{gray}{0.1}
	\definecolor{preto_ulisses}{gray}{0.1}
	\definecolor{green_ulisses}{gray}{0.1}
\fi


\def\codesize{\normalsize}

\lstdefinelanguage{HaskellUlisses} {
	basicstyle=\ttfamily\codesize,
	sensitive=true,
	morecomment=[l][\color{gray_ulisses}\ttfamily\codesize]{--},
	%% morecomment=[s][\color{gray_ulisses}\ttfamily\codesize]{\{-}{-\}},
	morestring=[b]",
	stringstyle=\color{haskellred},
	showstringspaces=false,
	numberstyle=\codesize,
	numberblanklines=true,
	showspaces=false,
	breaklines=true,
	showtabs=false,
	emph=
	{[1]
		FilePath,IOError,abs,acos,acosh,all,and,any,appendFile,approxRational,asTypeOf,asin,
		asinh,atan,atan2,atanh,basicIORun,break,catch,ceiling,chr,compare,concat,concatMap,
		const,cos,cosh,curry,cycle,decodeFloat,denominator,digitToInt,div,divMod,drop,
		dropWhile,either,elem,encodeFloat,enumFrom,enumFromThen,enumFromThenTo,enumFromTo,
		error,even,exp,exponent,fail,filter,flip,floatDigits,floatRadix,floatRange,floor,
		fmap,foldl,foldl1,foldr,foldr1,fromDouble,fromEnum,fromInt,fromInteger,
		fromRational,fst,gcd,getChar,getContents,getLine,head,id,inRange,index,init,intToDigit,
		interact,ioError,isAlpha,isAlphaNum,isAscii,isControl,isDenormalized,isDigit,isHexDigit,
		isIEEE,isInfinite,isLower,isNaN,isNegativeZero,isOctDigit,isPrint,isSpace,isUpper,iterate,
		last,lcm,length,lex,lexDigits,lexLitChar,lines,log,logBase,lookup,map,mapM,mapM_,max,
		maxBound,maximum,maybe,min,minBound,minimum,mod,negate,not,notElem,numerator,odd,
		or,pi,pred,primExitWith,print,product,properFraction,putChar,putStr,putStrLn,quot,
		quotRem,range,rangeSize,read,readDec,readFile,readFloat,readHex,readIO,readInt,readList,readLitChar,
		readLn,readOct,readParen,readSigned,reads,readsPrec,realToFrac,recip,rem,repeat,replicate,
		reverse,round,scaleFloat,scanl,scanl1,scanr,scanr1,seq,sequence,sequence_,show,showChar,showInt,
		showList,showLitChar,showParen,showSigned,showString,shows,showsPrec,significand,signum,sin,
		sinh,snd,span,splitAt,sqrt,subtract,succ,sum,tail,take,takeWhile,tan,tanh,threadToIOResult,toEnum,
		toInt,toInteger,toLower,toRational,toUpper,truncate,uncurry,undefined,unlines,until,unwords,unzip,
		unzip3,userError,words,writeFile,zip,zip3,zipWith,zipWith3,listArray,doParse,for,initTo,
        maxEvens,create,get,set,initialize,idVec,fastFib,fibMemo,
        insert,union,split,size,fromList,initUpto,trim,quickSort,insertSort,append,upperCase,
        copy, group, doDownLoop, mapAccumR, peekByteOff,
        pokeByteOff,spanByte, 
        good, bad, foo, explode, 
        fib, ack, 
        tLen,
        memcpy,writeChar,unsafeWrite,unsafeFreeze,
        singleton
	},
	emphstyle={[1]\color{haskellblue}},
	emph=
	{[2]
		Bool,Char,Double,Either,Float,IO,Integer,Int,Maybe,Ordering,Rational,Ratio,ReadS,ShowS,String,
		Word8,Nat,NonZero,Nat64,Text,ByteString,ByteStringSZ,ByteStringN,
        Ptr,ForeignPtr,CSize
        InPacket,Tree,Prop,TreeEq,TreeLt,Vec,
        NullTerm,IncrList,DecrList,UniqList,BST,MinHeap,MaxHeap,
        PtrN,ByteStringN,ByteStringEq,VO,ByteStringsEq,ByteStringNE
	},
	emphstyle={[2]\color{castanho_ulisses}},
	emph=
	{[3]
		case,class,data,deriving,do,else,if,import,in,infixl,infixr,instance,let,
		module,measure,predicate,of,primitive,then,refinement,type,where
	},
	emphstyle={[3]\color{preto_ulisses}\textbf},
	emph=
	{[4]
		quot,rem,div,mod,elem,notElem,seq
	},
	emphstyle={[4]\color{castanho_ulisses}\textbf},
	emph=
	{[5]
		PS,Tip,Node,EQ,False,GT,Just,LT,Left,Nothing,Right,True,Show,Eq,Ord,Num
	},
	emphstyle={[5]\color{green_ulisses}}
}

%%%ORIG
%%%\lstnewenvironment{code}
%%%{\textbf{Haskell Code} \hspace{1cm} \hrulefill \lstset{language=HaskellUlisses}}
%%%{\hrule\smallskip}

%V1
%\lstnewenvironment{code}
%{\smallskip \lstset{language=HaskellUlisses}}
%{\smallskip}

\lstnewenvironment{code}
{\lstset{language=HaskellUlisses}}
{}

\lstMakeShortInline[language=HaskellUlisses]@


\usepackage{amsthm}



\usepackage{color}
\usepackage{ifthen}

\newcommand{\isDecidable}{false} % true or false
\newcommand\restrictdecidable[2]{%
  \ifthenelse{\equal{\isDecidable}{true}}
    {{\color{green}{{#1}}}}
    {{\color{red}{{#2}}}}
\xspace}


\newtheorem{definition}{Definition}
\newtheorem{lemma}{Lemma}
\newtheorem{claim}{Claim}

% writting

%\newcommand\showproof[1]{#1}
\newcommand\showproof[1]{\texttt{proved}}
\newcommand\showprooftodo[1]{\texttt{TODO}}

%% RULE NAMES
\newcommand\rulename[1]{\textsc{#1}\xspace}


\newcommand\rwbase{\rulename{WF-Base}}
\newcommand\rwfun{\rulename{WF-Fun}}
\newcommand\rwcon{\rulename{WF-Con}}

\newcommand\rsubbase{\ensuremath{\preceq}\rulename{-Base}}
\newcommand\rsubfun{\ensuremath{\preceq}\rulename{-Fun}}
\newcommand\rsubcon{\ensuremath{\preceq}\rulename{-Con}}

\newcommand\rtvar{\rulename{T-Var}}
\newcommand\rtvarbase{\rulename{T-Var-Base}}
\newcommand\rtconst{\rulename{T-Const}}
\newcommand\rtsub{\rulename{T-Sub}}
\newcommand\rtexact{\rulename{T-Ex}}
\newcommand\rtcase{\rulename{T-Case}}
\newcommand\rtdata{\rulename{T-Data}}
\newcommand\rtfun{\rulename{T-Fun}}
\newcommand\rtapp{\rulename{T-App}}
\newcommand\rtlet{\rulename{T-Let}}

\newcommand\rimpl{\ensuremath{\Rightarrow}\rulename{-Base}}

\newcommand\dcti{\dct{i}}
\newcommand\dct[1]{\ensuremath{D^T_{#1}}}
%% EXPRESSIONS
\newcommand\efix[1]{\ensuremath{\mathtt{fix}_{#1}\xspace}}
\newcommand\efun[2]{\ensuremath{\lambda #1 . #2}}
\newcommand\eapp[2]{\ensuremath{#1 \ #2}}
\newcommand\edata[2]{\ensuremath{#1 \ #2}}
\newcommand\ecase[5]{\ensuremath{\text{case}_T\ #1\ #2\ \text{of}\ \overline{\edata{#3}{#4} \rightarrow #5}}}
\newcommand\elet[3]{\ensuremath{\text{let}\ #1 = #2\ \text{in}\ #3}}
\newcommand\erec[3]{\ensuremath{\mu #1 \lambda #2 . #3}}

\newcommand\etrue{\ensuremath{\text{true}}}
\newcommand\efalse{\ensuremath{\text{false}}}


%% TYPES
\newcommand\tint{\ensuremath{\text{int}}}
\newcommand\tbool{\ensuremath{\text{bool}}}
\newcommand\tref[3]{\ensuremath{\left\lbrace {#1} \colon {#2} \mid {#3} \right\rbrace}}
\newcommand\tconref[4]{\tref{#3}{\tcon{#1}{#2}}{#4}}
\newcommand\tcon[2]{\ensuremath{#1\ #2}}
\newcommand\tfun[3]{\ensuremath{#1\colon #2 \rightarrow #3}}
\newcommand\tfunref[5]{\tfun{#1}{#2}{#3}}
% \newcommand\tfunref[5]{\tref{#4}{(\tfun{#1}{#2}{#3})}{#5}}

%% OTHER

\newcommand\crash{\ensuremath{\mathtt{crash}}}
\newcommand\eqb[1]{\ensuremath{=_{#1}}}
\newcommand\eq{\eqb{}}

\newcommand\validn[2]{\ensuremath{\text{Valid}_{#2}\ (#1)}}
\newcommand\finn[2]{\ensuremath{\text{Fin}_{#2}\ (#1)}}
\newcommand\validi[1]{\ensuremath{\text{Valid}_{i}\ (#1)}}
\newcommand\fini[1]{\ensuremath{\exists v. \evals{#1}{v}}}
\newcommand\fin[1]{\ensuremath{\exists v. \evals{#1}{v}}}
%\newcommand\fini[1]{\ensuremath{\text{Fin}_{i}\ (#1)}}
%\newcommand\fin[1]{\ensuremath{\text{Fin}\ (#1)}}

\newcommand\valid[1]{\ensuremath{\text{Valid}(#1)}}

% \newcommand\generalconditionInterp[2]{\ensuremath{\fin{#1} \Rightarrow \valid{#2}}}
% \newcommand\generalconditionImpl[2]{\ensuremath{\valid{#1} \Rightarrow \valid{#2}}}
% \newcommand\generalconditionImplOne[2]{\ensuremath{\valid{#1}}}

\newcommand\generalconditionInterp[2]
	{\ensuremath{(\forall i. \fini{#1} \Rightarrow \validi{#2})}}
\newcommand\generalconditionImpl[2]
	{\ensuremath{\forall i . \validi{#1} \Rightarrow \validi{#2}}}
\newcommand\generalconditionImplOne[1]
	{\ensuremath{\forall i . \validi{#1}}}

%% general condition interp can be any condition in which 
%% the second argument appears in positive places

\newcommand\tforalli[1]{\ensuremath{\forall (1\leq i\leq i_{#1})}}

%OTHER
\newcommand\ty[1]{\ensuremath{\text{ty}({#1})}}
\newcommand\sub[2]{\ensuremath{\left[ #2 / #1 \right]}}
\newcommand\erase[1]{\ensuremath{\lfloor #1 \rfloor}}

\newcommand\interp[1]{\ensuremath{[|#1|]}}
\newcommand\eval[2]{\ensuremath{#1 \hookrightarrow #2}}
\newcommand\evals[2]{\ensuremath{#1 \hookrightarrow^\star #2}}
\newcommand\evalsi[3]{\ensuremath{#1 \hookrightarrow^{#3} #2}}
\newcommand\hastype[3]{\ensuremath{#1 \vdash #2 : #3}}
\newcommand\ispure[2]{\ensuremath{#1 \vdash_{\text{pure}} #2}}
\newcommand\hastypebase[3]{\ensuremath{#1 \vdash_B #2 : #3}}
\newcommand\shastype[3]{\ensuremath{#1 \vdash #2 \in #3}}
\newcommand\iswellformed[2]{\ensuremath{#1 \vdash #2}}
\newcommand\biswellformed[2]{\ensuremath{#1 \vdash_B #2}}
\newcommand\issubtype[3]{\ensuremath{#1 \vdash #2 \preceq #3}}
\newcommand\sissubtype[3]{\ensuremath{#1 \vdash #2 \subseteq #3}}
\newcommand\issubref[3]{\ensuremath{#1 \vdash #2 \Rightarrow #3}}
\title{Refinements for Lazy Evaluation}
\begin{document}

\section{Language}
\subsection{Syntax}
\input{language}
\begin{itemize}
\item $\texttt{Bool}\in T$, $i_\texttt{Bool} = 2$
\item $\texttt{True}  \equiv D^{\texttt{Bool}}_1, \texttt{False} \equiv D^{\texttt{Bool}}_2$
\item
$$\text{if}\ e\ \text{then}\ e_1\ \text{else}\ e_2 \doteq 
\text{case}_{\texttt{Bool}}\ e\ x\ \text{of}\ \{\texttt{True}\Rightarrow e_1;\texttt{False}\Rightarrow e_2 \}
$$
\end{itemize}
\subsection{Operational Semantics}
\input{operational}

\section{Undecidable System}
\subsection{Erasing}
\begin{align*}
\erase{\tref{v}{b}{e}}&=b\\
\erase{\tfunref{x}{\tau_x}{\tau}{v}{e}}&= \erase{\tau_x} \rightarrow \erase{\tau}\\
% \erase{\tconref{T}{\overline{\tau}}{v}{e}}&= \tcon{T}\overline{\erase{\tau}}
\end{align*}

\begin{align*}
\erase{\emptyset}&=\emptyset\\
\erase{x\colon\tau, \Gamma}&= x\colon\erase{\tau},\erase{\Gamma}
\end{align*}

\subsection{Substitutions}
\begin{align*}
(\tref{v}{b}{e})\sub{y}{e_y} &=\tref{v}{b}{e\sub{y}{e_y}}\\
(\tfunref{x}{\tau_x}{\tau}{v}{e})\sub{y}{e_y} &=
	\tfunref{x}{(\tau_x\sub{y}{e_y})}{(\tau\sub{y}{e_y})}{v}{e\sub{y}{e_y}}\\
% (\tconref{T}{\overline{\tau}}{v}{e})\sub{y}{e_y} 
% 	&=\tconref{T}{\overline{\tau\sub{y}{e_y}}}{v}{e\sub{y}{e_y}}
\end{align*}

\subsection{Interpretations}
\begin{definition} \label{def:valid}
Let \validi{\star} be predicates on expressions such that
\begin{enumerate}
\item For any $x, e, e_r, \theta$, if \eval{e}{e'} then 
	\generalconditionImpl{\theta\ e_r\sub{x}{e'}}{\theta\ e_r\sub{x}{e}} and
	\generalconditionImpl{\theta\ e_r\sub{x}{e}}{\theta\ e_r\sub{x}{e'}}.

\item For any $e_1, e_2$, 
$$\validi{e_1} \land \validi{e_2} \Rightarrow \validi{e_1 \land e_2}$$

\item $$\validi{\etrue}$$
\end{enumerate} 
\end{definition}

$$
\begin{array}{lll}
\interp{\tref{v}{b'}{e_v}} &=
	\{e \mid & \hastype{}{e}{b} 
	\land 
	\generalconditionInterp{e}{e_v\sub{v}{e}} 
	\}\\
%%\interp{\tref{v}{T}{e_T}} &=
%%	\{e \mid & \hastype{}{e}{b} 
%%	\land 
%%	\generalconditionInterp{e}{e_T\sub{v}{e}} 
%%	\}\\
%%	&\cap\
%%	\{ e \mid & \tforalli{T} \{
%%		\dcti \in \interp{\overline{x\colon\tau_{\dcti}} \rightarrow \tref{v}{T}{e'_T}}\\
%%		&& \land\ \theta = \sub{x}{e_{y_i}} \land \forall e_{y_i} \in \interp{\theta\ t_{\dcti}} \\
%%		&& e \in \interp{\tref{v}{T}{\theta e'_T}} 
%%				\Rightarrow e_i\sub{x}{e}\sub{y_i}{e_{y_i}} \in \interp{\tau}	
%%	\} \\
%%	&&\Rightarrow \ecase{e}{x}{\dcti}{\overline{y_i}}{e_i} \in \interp{\tau}
%%	\}
%%	\\

\interp{\tref{v}{T}{e_T}} &=
	\{e \mid & \hastype{}{e}{T} 
	\land 
	\generalconditionInterp{e}{e_T\sub{v}{e}}\\
		&&\land\ \evals{e}{v_e} \Rightarrow \{v_e = \dcti\ \overline{e_{y_i}} \land 
		\dcti \in \interp{\overline{x\colon\tau_{\dcti}} \rightarrow \tref{v}{T}{e'_T}}\\
		&& \land\ \theta = \sub{x}{e_{y_i}} \land e_{y_i} \in \interp{\theta\ t_{\dcti}} \land
		 e \in \interp{\tref{v}{T}{\theta e'_T}}\} 
	\} \\

\interp{\tfunref{x}{\tau_x}{\tau}{v}{e_v}} &=
	\{e \mid & \hastype{}{e}{\erase{\tau_x} \rightarrow \erase{\tau}} 
	\land 
%% TFUNREF 	\generalconditionInterp{e}{e_v\sub{v}{e}} \\
%% TFUNREF 	&& \land \ 
	\forall e_x \in \interp{\tau_x}. \
	 \eapp{e}{e_x} \in \interp{\tau\sub{x}{e_x}} 
	 \}\\
%%\interp{\tconref{T}{\overline{\tau}}{v}{e_v}} &= 
%%	\{e \mid & \hastype{}{e}{\erase{\tconref{T}{\overline{\tau}}{v}{e_v}}} 
%%	\land 
%%	\generalconditionInterp{e}{e_v\sub{v}{e}}
%%	\land \\&& 
%%	\evals{e}{\edata{D}{\overline{e}}} \Rightarrow 
%%	\ty{D} = \overline{x\colon\tau'}\rightarrow\tconref{T}{\overline{\tau}}{v}{e_T} \land 	
%%	\overline{e_i \in \interp{\overline{\sub{x_i}{e_i}} \tau'_i}}
%%	 \}
\end{array}
$$

\subsection{Typing}
\hfill\mbox{\hastype{\Gamma}{e}{\tau}}
%\rtvar
%\rtvarbase
%\rtconst
%\rtsub
%\rtexact
%\rtfun
%\rtapp
%\rtcase
%\rtlet
$$
\inference{
	\hastype{\Gamma}{e}{\tref{v}{b}{e'}}
}{
	\hastype{\Gamma}{e}{\tref{v}{b}{v \eqb{b} e}}
}[\rtexact]
$$
$$
\inference{
	(x,\tref{v}{b}{e}) \in \Gamma
}{
	\hastype{\Gamma}{x}{\tref{v}{b}{v \eqb{b} x}}
}[\rtvarbase]
\qquad
\inference{
	(x,\tau) \in \Gamma &&
	\tau \neq (x,\tref{v}{b}{e})
}{
	\hastype{\Gamma}{x}{\tau}
}[\rtvar]
$$
$$
\inference{
}{
	\hastype{\Gamma}{w}{\ty{w}}
}[\rtconst]
\qquad
\inference{
	\hastype{\Gamma}{e}{\tau'} &&
	\issubtype{\Gamma}{\tau'}{\tau} &&
	\iswellformed{\Gamma}{\tau} &&
}{
	\hastype{\Gamma}{e}{\tau}
}[\rtsub]
$$
$$
\inference{
	\hastype{\Gamma, x\colon\tau_x}{e}{\tau} &&
	\iswellformed{\Gamma}{\tau_x}
}{
	\hastype{\Gamma}{\efun{x}{e}}{(\tfun{x}{\tau_x}{\tau})}
}[\rtfun]
\qquad
\inference{
	\hastype{\Gamma}{e_1}{(\tfunref{x}{\tau_{x}}{\tau}{v}{e_v})} &&
	\hastype{\Gamma}{e_2}{\tau_{x}}
}{
	\hastype{\Gamma}{\eapp{e_1}{e_2}}{\tau\sub{x}{e_2}}
}[\rtapp]
$$
$$
\inference{
	\hastype{\Gamma}{e_x}{\tau_{x}} &&
	\hastype{\Gamma,x\colon\tau_x}{e_2}{\tau} &&
	\iswellformed{\Gamma}{\tau}
}{
	\hastype{\Gamma}{\elet{x}{e_x}{e}}{\tau}
}[\rtlet]
$$
$$
\inference{
	\hastype{\Gamma}{e}{\tref{v}{T}{e_T}} && \iswellformed{\Gamma}{\tau}\\
	\tforalli{T}. \left\lbrace
	\begin{array}{ll}
 	\ty{\dcti} = \overline{x\colon\tau_{\dcti}}\rightarrow \tref{v}{T}{e'_T} \qquad
 	\theta = \sub{x}{y_i} \cr
	\hastype{\Gamma, \overline{y_i\colon\theta\ \tau_{\dcti}}, 
				x\colon \tref{v}{T}{e_T \land \theta e'_T}
				}{e_i}{\tau}
	\end{array} \right.
}{
	\hastype{\Gamma}{\ecase{e}{x}{\dcti}{\overline{y}_i}{e_i}}{\tau}
}[\rtcase]
$$

\hfill\mbox{\iswellformed{\Gamma}{\tau}}
$$
\inference{
	\hastypebase{\erase{\Gamma}, v\colon b}{e}{\tbool}
}{
	\iswellformed{\Gamma}{\tref{v}{b}{e}}
}[\rwbase]
\qquad
\inference{
%%TFUNREF	\hastypebase{\erase{\Gamma}, v\colon b}{e}{\tbool} &&
	\iswellformed{\Gamma}{\tau_x} &&
	\iswellformed{\Gamma, x \colon \tau_x}{\tau}
}{
	\iswellformed{\Gamma}{\tfunref{x}{\tau_x}{\tau}{v}{e}}
}[\rwfun]
$$

\hfill\mbox{\issubtype{\Gamma}{\tau}{\tau}}
$$
\inference{
	\issubref{\Gamma, v:b}{e_1}{e_2}
}{
	\issubtype{\Gamma}{\tref{v}{b}{e_1}}{\tref{v}{b}{e_2}}
}[\rsubbase]
\qquad
\inference{
%% FUNREF	\issubref{\Gamma, v:\tfun{x}{\tau_x}{\tau}}{e_1}{e_2} \\
	\issubtype{\Gamma}{\tau'_x}{\tau_x} &&
	\issubtype{\Gamma, x \colon \tau'_x}{\tau}{\tau'}
}{
	\issubtype{\Gamma}{\tfunref{x}{\tau_x}{\tau}{v}{e_1}}{\tfunref{x}{\tau'_x}{\tau'}{v}{e_2}}
}[\rsubfun]
$$

\hfill\mbox{\issubref{\Gamma}{e}{e}}
$$
\inference{
	\forall \theta. \iswellformed{\Gamma}{\theta} \land
	\generalconditionImpl{\theta\ e_1}{\theta\ e_2}
}{
	\issubref{\Gamma}{e_1}{e_2}
}[\rimpl]
$$

\hfill\mbox{\iswellformed{}{\Gamma}}
$$
\inference{
	\iswellformed{}{\Gamma} &&
	\iswellformed{\Gamma}{\tau}
}{
	\iswellformed{}{x\colon\tau,\Gamma}
}
\qquad
\inference{}{\iswellformed{}{\emptyset}}
$$

\hfill\mbox{\iswellformed{\Gamma}{\theta}}
$$
\inference{
	\forall x \in \text{Dom}(\Gamma). 
	\theta(x) \in \interp{\theta \ \Gamma(x)}
}{
	\iswellformed{\Gamma}{\theta}
}
$$
\subsection{Constants and Data Constructors}
\begin{definition}\label{def:constants}
\crash is an untyped constant.

For each constant or data constructor $w \neq \crash$

\newcommand\const{\ensuremath{w}}
\begin{enumerate}
\item \hastype{\emptyset}{\const}{\ty{\const}} and \iswellformed{}{\ty{\const}}
\item If $\ty{\const} = \tfun{x}{\tau_x}{\tau}$, then for each $v$ 
	$\interp{\const}(v)$ is defined and 
	if \hastype{\emptyset}{v}{\tau_x} then
	\hastype{}{\interp{\const}(v)}{\tau\sub{x}{v}},
	otherwise  $\interp{\const}(v) = \crash$
	

\item If $\ty{\const} = \tref{v}{b}{e}$, 
then 
\validi{{e\sub{v}{\const}}} and 
$\forall \const'\, \const' \neq \const. \lnot (\validi{e\sub{v}{\const'}})$ 
%
% \generalconditionInterp{\const}{e\sub{v}{\const}} and 
% $\forall \const'\ \const' \neq \const. \lnot (\generalconditionInterp{\const}{e\sub{v}{\const'}})$ 
\end{enumerate}
Moreover, for any base type $b$, \eqb{b} is a constant and 
\begin{itemize}
\item For any expression $e$ we have 
$$\generalconditionImplOne{e \eqb{b} e}$$
\item For any base type $b$ 
$$\ty{=_b} \equiv \tfun{x}{b}{\tfun{y}{b}{\tbool}}$$
\end{itemize}

For each $T$ there are exactly $i_T$ constants with result type $\tref{v}{T}{e_T}$, 
namely   $\dcti$, $\forall 1 \leq i \leq i_T$.
\end{definition}

\subsection{Semantic Typing}
\begin{align*}
\shastype{\Gamma}{e}{\tau} & \doteq
	\forall \theta . \iswellformed{\Gamma}{\theta} \Rightarrow \theta\ e \in \interp{\theta \ \tau}\\
\sissubtype{\Gamma}{\tau_1}{\tau_2} & \doteq 
	\forall \theta . \iswellformed{\Gamma}{\theta} \Rightarrow \interp{\theta\ \tau_1} \subseteq \interp{\theta\ \tau_2}
\end{align*}

\begin{lemma}\label{lemma:sevals}
If $e$ diverges and \hastypebase{\emptyset}{e}{\erase{\tau}} 
then \shastype{\emptyset}{e}{\tau}
\end{lemma}
\showprooftodo{
}


\begin{lemma}\label{lemma:sevals}
If \evals{e}{e'} and ${e'}\in{\interp{\tau}}$ 
then ${e}\in{\interp{\tau}}$
\end{lemma}
\showprooftodo{
	\begin{proof}
	\input{proofs/sevals}
	\end{proof}
}

\begin{lemma}.
\begin{enumerate}
\item If \issubtype{\Gamma}{\tau_1}{\tau_2} then \sissubtype{\Gamma}{\tau_1}{\tau_2} 
\item If \hastype{\Gamma}{e}{\tau} then \shastype{\Gamma}{e}{\tau} 
\end{enumerate}
\end{lemma}
\showprooftodo{
	\begin{proof}
	\begin{enumerate}
\item \label{proof:ssub} Assume \issubtype{\Gamma}{\tau_1}{\tau_2} 
We will prove it by induction on the derivation tree:
\begin{itemize}
\item\rsubbase. We have
$$\issubtype{\Gamma}{\tref{v}{b}{e_1}}{\tref{v}{b}{e_2}}$$
By inversion we get 
$$\issubref{\Gamma, v\colon b}{e_1}{e_2}$$
By inversion of \rimpl we have
$$	\forall \theta. \iswellformed{\Gamma, v\colon b}{\theta} \land
	\generalconditionImpl{\theta\ e_1}{\theta\ e_2}
(1)$$

We want to prove 
$$\sissubtype{\Gamma}{\tref{v}{b}{e_1}}{\tref{v}{b}{e_2}}$$
Equivalently
$$	
	\forall \theta . \iswellformed{\Gamma}{\theta} \Rightarrow 
	\interp{\theta\ \tref{v}{b}{e_1}} \subseteq \interp{\theta\ \tref{v}{b}{e_2}}
$$
Equivalently
\begin{align*}
	\forall \theta . \iswellformed{\Gamma}{\theta} \\& \Rightarrow 
		\{e \mid \hastype{}{e}{b} 
 			\land 
			\generalconditionInterp{e}{\theta\ e_1\sub{v}{e}} 
		\}	
	\\& \subseteq 
		\{e \mid \hastype{}{e}{b} 
			\land 
			\generalconditionInterp{e}{\theta\ e_2\sub{v}{e}}
		 \}	
\end{align*}
Since $e \in \interp{b}$, we have \iswellformed{\Gamma,v\colon b}{\theta,\sub{v}{e}}.
So, from $(1)$ for $\theta := \theta,\sub{v}{e}$
we have 
$$	
	\generalconditionImpl{\theta\ e_1\sub{v}{e}}{\theta\ e_2\sub{v}{e}}
$$

\item\rsubfun Assume
$$\issubtype{\Gamma}{x:\tau_x \rightarrow \tau}{x:\tau'_x \rightarrow \tau'}$$
By inversion we have
$$	\issubtype{\Gamma}{\tau'_x}{\tau_x} \qquad
	\issubtype{\Gamma, x \colon \tau'_x}{\tau}{\tau'}
$$
By IH
$$	\sissubtype{\Gamma}{\tau'_x}{\tau_x} (1) \qquad
	\sissubtype{\Gamma, x \colon \tau'_x}{\tau}{\tau'} (2)
$$
We want to show that 
$$\sissubtype{\Gamma}{x:\tau_x \rightarrow \tau}{x:\tau'_x \rightarrow \tau'}$$
Equivalently
$$	
	\forall \theta . \iswellformed{\Gamma}{\theta} \Rightarrow 
	\interp{\theta\ x:\tau_x \rightarrow \tau} 
	\subseteq 
	\interp{\theta\ x:\tau'_x \rightarrow \tau'}
$$
Equivalently
\begin{align*}
	\forall \theta &. \iswellformed{\Gamma}{\theta} \\&\Rightarrow 
	\{e \mid \hastype{}{e}{\erase{\tau_x} \rightarrow \erase{\tau}} 
	\land 
	\forall e_x \in \interp{\tau_x}. \
	 \eapp{e}{e_x} \in \interp{\tau\sub{x}{e_x}} 
	 \}\\ &
	\subseteq 
	\{e \mid \hastype{}{e}{\erase{\tau'_x} \rightarrow \erase{\tau'}} 
	\land 
	\forall e_x \in \interp{\tau'_x}. \
	 \eapp{e}{e_x} \in \interp{\tau'\sub{x}{e_x}} 
	 \}
\end{align*}
The above holds, as for any $e, e_x$
if $e_x \in \interp{\tau'}$
then by $(1)$
$e_x \in \interp{\tau}$.
Also, by $(2)$
if $\eapp{e}{e_x} \in \interp{\tau\sub{x}{e_x}}$
then
$\eapp{e}{e_x} \in \interp{\tau'\sub{x}{e_x}}$.
\end{itemize}
 
\item Assume \hastype{\Gamma}{e}{\tau}. 
We will prove it by induction on the derivation tree.

\begin{itemize}
\item \rtexact Assume
$$\hastype{\Gamma}{e}{\tau}$$
where $\tau \equiv \tref{v}{b}{v \eqb{b} e}$.
By inversion we have
$$\hastype{\Gamma}{e}{\tref{v}{b}{e'}}$$
We need to show that 
$$	\forall \theta . \iswellformed{\Gamma}{\theta} \Rightarrow \theta\ e \in \interp{\theta \ \tau}$$
Which holds, as by definition of \eqb{b}
$\generalconditionImplOne{(v \eqb{b} \theta\ e)\sub{v}{\theta\ e}}$
\item\rtvar Assume
$$	\hastype{\Gamma}{e}{\tau}$$
where $e \equiv x$
By inversion we have
$$(x,\tau) \in \Gamma$$
We need to show that 
$$	\forall \theta . \iswellformed{\Gamma}{\theta} \Rightarrow \theta\ x \in \interp{\theta \ \tau}$$
Which holds by the definition of well-formed substitutions
\item\rtvarbase Assume
$$\hastype{\Gamma}{e}{\tau}$$
where $e\equiv x$ and $\tau\equiv\tref{v}{b}{v\eqb{b}x}$.
By inversion
$$(x,\tref{v}{b}{e_r}) \in \Gamma$$ 
We need to show that 
$$	\forall \theta . \iswellformed{\Gamma}{\theta} \Rightarrow \theta\ x \in \interp{\theta \ \tau}$$
Equivalently that 
$$\forall e.
 e\in\interp{\tref{v}{b}{e_r}} \Rightarrow e \in \interp{\tref{v}{b}{v \eqb{b} e}}$$
which holds, as by the definition of \eqb{b}
$$\generalconditionImplOne{e\eqb{b}e}$$
\item\rtconst. Assume
$$\hastype{\Gamma}{e}{\tau}$$
where $e \equiv w$  and $\tau\equiv\ty{c}$.
Then \shastype{\Gamma}{e}{\tau} holds by Definition \ref{def:constants}.
\item \rtcase It follows from the definition of \interp{\tref{v}{T}{e}}
using that $$\validi{e_T} \land \validi{theta e'_T} \Rightarrow \validi{e_T \land \theta e'_T}$$
to prove that $e \in {\tref{v}{T}{e_T \land theta e'_T}}$
\item\rtsub Assume 
$$	\hastype{\Gamma}{e}{\tau}$$
By inversion
$$
	\hastype{\Gamma}{e}{\tau'}\ (1) \qquad
	\issubtype{\Gamma}{\tau'}{\tau}\ (2) \qquad
	\iswellformed{\Gamma}{\tau}\ (3)
$$
By IH on $(1)$ we have
$$	\shastype{\Gamma}{e}{\tau'}\ (4)$$
By \ref{proof:ssub} on $(2)$ we have
$$	\sissubtype{\Gamma}{\tau'}{\tau}\ (5)$$
By $(4)$ and $(5)$ we get
$$	\shastype{\Gamma}{e}{\tau}$$
\item\rtfun Assume
$$	\hastype{\Gamma}{e}{\tau}$$
where $e \equiv \efun{x}{e'}$ and 
$\tau \equiv\tfun{x}{\tau'_x}{\tau'}$.
By inversion we get
$$
	\hastype{\Gamma, x\colon\tau'_x}{e'}{\tau'}\ (1) \qquad
	\iswellformed{\Gamma}{\tau'_x}\ (2)
$$
By IH on $(1)$ we have
$$
	\shastype{\Gamma, x\colon\tau'_x}{e'}{\tau'}\ (3)
$$
Equivalently
$$	
\forall \theta . \iswellformed{(\Gamma,x\colon\tau'_x)}{(\theta\sub{x}{e_x})} 
	\Rightarrow (\theta\sub{x}{e_x})\ e' \in \interp{(\theta\sub{x}{e_x}) \ \tau'}\\
$$
Or
$$	
\forall \theta . \iswellformed{\Gamma}{\theta} \Rightarrow
\forall e_x . e_x \in \interp{\tau'_x} \Rightarrow
	\theta\ \eapp{e}{e_x} \in \interp{\theta \ (\tau'\sub{x}{e_x})}\\
$$
Moreover, $\hastype{e}{\erase{\tau'_x}\rightarrow{\erase{\tau}}}$.
So,
$$	
\forall \theta . \iswellformed{\Gamma}{\theta}\ \theta\ e\in \interp{\theta\ \tau}
$$
Or, $$\shastype{\Gamma}{e}{\tau}$$
\item\rtapp
Assume
$$\hastype{\Gamma}{e}{\tau}$$
where $e\equiv\eapp{e_1}{e_2}$ and $\tau\equiv\tau'\sub{x}{e_2}$.
By inversion:
$$
	\hastype{\Gamma}{e_1}{(\tfun{x}{\tau'_{x}}{\tau'})}\ (1)\qquad
	\hastype{\Gamma}{e_2}{\tau'_{x}}\ (2)
$$
By IH we get
$$
	\shastype{\Gamma}{e_1}{(\tfun{x}{\tau'_{x}}{\tau'})}\ (3)\qquad
	\shastype{\Gamma}{e_2}{\tau'_{x}}\ (4)
$$
So 
$$\forall \theta. \iswellformed{\Gamma}{\theta}\Rightarrow
\forall e_x \in \interp{\theta\ \tau'_x} \Rightarrow
	\eapp{(\theta e_1)}{e_x} \in \interp{\theta\ \tau'\sub{x}{e_x}}
\ (5)$$
and
$$\forall \theta. \iswellformed{\Gamma}{\theta}\Rightarrow
	\theta\ e_2 \in \interp{\theta\ \tau'_x}
\ (6)$$
From $(5)$ and $(6)$, we get
$$\forall \theta. \iswellformed{\Gamma}{\theta}\Rightarrow
	\theta\ e \in \interp{\theta\ \tau}
$$
Or $$\shastype{\Gamma}{e}{\tau}$$
\end{itemize}
\end{enumerate}

	\end{proof}
}


\begin{lemma}[Substitution]\label{lemma:substitution}
If \shastype{\Gamma}{e_x}{\tau_x} and \iswellformed{}{\Gamma, x\colon\tau_x ,\Gamma'}, then 
\begin{enumerate}
\item If 
	\issubtype{\Gamma, x\colon\tau_x, \Gamma'}{\tau_1}{\tau_2}
	then
	\issubtype{\Gamma, \sub{x}{e_x}\Gamma'}{\sub{x}{e_x}\tau_1}{\sub{x}{e_x}\tau_2}
\item If 
	\hastype{\Gamma, x\colon\tau_x, \Gamma'}{e}{\tau}
	then
	\hastype{\Gamma, \sub{x}{e_x}\Gamma'}{\sub{x}{e_x}e}{\sub{x}{e_x}\tau}
\item If 
	\iswellformed{\Gamma, x\colon\tau_x, \Gamma'}{\tau}
	then
	\iswellformed{\Gamma, \sub{x}{e_x}\Gamma'}{\sub{x}{e_x}\tau}
\end{enumerate}
\end{lemma}
\showproof{
	\begin{proof}
	If \hastype{\Gamma}{e_x}{\tau_x} and \iswellformed{\Gamma, x\colon\tau_x ,\Gamma'}, then 
\begin{enumerate}
\item\label{proof:sub:sub} Assume
	$$\issubtype{\Gamma, x\colon\tau_x, \Gamma'}{\tau_1}{\tau_2}$$
We will prove the lemma by induction on the derivation tree.
\begin{itemize}
\item \rsubbase
Assume
	$$\issubtype{\Gamma, x\colon\tau_x, \Gamma'}{\tau_1}{\tau_2}$$
where $\tau_1 \equiv \tref{v}{b}{e_1}$
and   $\tau_2 \equiv \tref{v}{b}{e_2}$
By inversion
	$$
	\issubref{\Gamma, x\colon\tau_x, \Gamma',v:b}{e_1}{e_2}
	$$
By inversion
	\begin{align*}
	\forall &\theta, e'_x, \theta',e .
	\iswellformed{\Gamma, x\colon\tau_x, \Gamma',v:b}{\theta\sub{x}{e'_x}\theta'\sub{v}{e}}\\& \Rightarrow
	\generalconditionImpl{(\theta\sub{x}{e'_x}\theta'\sub{v}{e})e_1}{(\theta\sub{x}{e'_x}\theta'\sub{v}{e})e_2}
	\end{align*}

Since \shastype{\Gamma}{e_x}{\tau_x}, so
	\begin{align*}
	\forall &\theta, \theta',e .
	\iswellformed{\Gamma,x\colon\tau_x, \Gamma',v:b}{\theta \sub{x}{e_x}\theta'\sub{v}{e}}\\& \Rightarrow
	\generalconditionImpl{(\theta\sub{x}{e_x}\theta'\sub{v}{e})e_1}{(\theta\sub{x}{e_x}\theta'\sub{v}{e})e_2}
	\end{align*}
Since \shastype{\Gamma}{e_x}{\tau_x}, so
	\begin{align*}
	\forall &\theta, \theta',e .
	\iswellformed{\Gamma,\sub{x}{e_x}\Gamma',v:b}{\theta\theta'\sub{v}{e}}\\& \Rightarrow
	\generalconditionImpl{(\theta\theta'\sub{v}{e})(e_1\sub{x}{e_x})}
						 {(\theta\theta'\sub{v}{e})(e_2\sub{x}{e_x})}
	\end{align*}
So,
	$$
	\issubref{\Gamma, \sub{x}{e_x}\Gamma',v:b}{e_1\sub{x}{e_x}}{e_2\sub{x}{e_x}}
	$$
And
	$$
	\issubtype{\Gamma, \sub{x}{e_x}\Gamma',v:b}{t_1\sub{x}{e_x}}{t_2\sub{x}{e_x}}
	$$
\item \rsubfun
Assume
	$$\issubtype{\Gamma, x\colon\tau_x, \Gamma'}{\tau_1}{\tau_2}$$
where $\tau_1 \equiv \tfun{y}{\tau_y}{\tau}$
and   $\tau_2 \equiv \tfun{y}{\tau'_y}{\tau'}$
By inversion
	$$
	\issubtype{\Gamma, x\colon\tau_x, \Gamma'}{\tau'_y}{\tau_y}\ (1) \qquad
	\issubtype{\Gamma, x\colon\tau_x, \Gamma',y\colon\tau'_y}{\tau}{\tau'}\ (2)
	$$
By IH	
	$$
	\issubtype{\Gamma, \sub{x}{e_x}\Gamma'}{\tau'_y\sub{x}{e_x}}{\tau_y\sub{x}{e_x}} \qquad
	\issubtype{\Gamma, \sub{x}{e_x}\Gamma',y\colon\tau'_y\sub{x}{e_x}}{\tau\sub{x}{e_x}}{\tau'\sub{x}{e_x}}
	$$
By rule \rsubfun	
	$$
	\issubtype{\Gamma, \sub{x}{e_x}\Gamma'}{\tau_1\sub{x}{e_x}}{\tau_2\sub{x}{e_x}}
	$$
\end{itemize}

$$
\inference{
	\issubref{\Gamma, v:b}{e_1}{e_2}
}{
	\issubtype{\Gamma}{\tref{v}{b}{e_1}}{\tref{v}{b}{e_2}}
}[\rsubbase]
\qquad
\inference{
	\issubtype{\Gamma}{\tau'_x}{\tau_x} &&
	\issubtype{\Gamma, x \colon \tau'_x}{\tau}{\tau'}
}{
	\issubtype{\Gamma}{x:\tau_x \rightarrow \tau}{x:\tau'_x \rightarrow \tau'}
}[\rsubfun]
$$

	
	then
	\issubtype{\Gamma, \sub{x}{e_x}\Gamma'}{\sub{x}{e_x}\tau_1}{\sub{x}{e_x}\tau_2}
\item \label{proof:sub:type} 
Assume 
	\hastype{\Gamma, x\colon\tau_x, \Gamma'}{e}{\tau}.
We will prove the lemma by induction on the derivation tree.
\begin{itemize}
\item \rtexact Assume
	$$\hastype{\Gamma, x\colon\tau_x, \Gamma'}{e}{\tau}$$
where $\tau \equiv \tref{v}{b}{v \eqb{b} e}$.
By inversion we get
	$$\hastype{\Gamma, x\colon\tau_x, \Gamma'}{e}{\tref{v}{b}{e'}}$$
By IH
	$$\hastype{\Gamma, \sub{x}{e_x} \Gamma'}{e\sub{x}{e_x}}{\tref{v}{b}{e'\sub{x}{e_x}}}$$
By rule \rtexact
	$$\hastype{\Gamma, \sub{x}{e_x} \Gamma'}{e\sub{x}{e_x}}{\tref{v}{b}{v \eqb{b} \sub{x}{e_x}}}$$
Or
	$$\hastype{\Gamma, \sub{x}{e_x} \Gamma'}{e\sub{x}{e_x}}{\tau\sub{x}{e_x}}$$
\item \rtvar Assume 
	$$\hastype{\Gamma, x\colon\tau'_x, \Gamma'}{e}{\tau}$$
where $e \equiv y$.
By inversion 
$$(y,\tau )\in \Gamma, x\colon\tau'_x, \Gamma'$$
Assume
$$(y,\tau)\in \Gamma$$
By rule \rtvar
$$\hastype{\Gamma,\sub{x}{e_x}\Gamma'}{e}{\tau}$$
Since \iswellformed{}{\Gamma}, $x$ cannot appear in $\tau$
so $\tau\sub{x}{e_x}\equiv\tau$.
Also, $x\neq y$, so $e\sub{x}{e_x}\equiv e$.
So,
$$\hastype{\Gamma,\sub{x}{e_x}\Gamma'}{e\sub{x}{e_x}}{\tau\sub{x}{e_x}}$$

Assume
$$y \equiv x$$
By lemma's assumption 
$$\shastype{\Gamma}{e_x}{\tau_x}$$
so
$$\hastype{\Gamma,\sub{x}{e_x}\Gamma'}{e_x}{\tau'_x}$$
Since $x = y$, $e\sub{x}{e_x} \equiv e_x$.
Also, since $x \notin Dom(\Gamma)$ 
it cannot appear in $\tau'_x$,so
$\tau\sub{x}{e_x} \equiv \tau \equiv \tau'_x$.
So,
$$\hastype{\Gamma,\sub{x}{e_x}\Gamma'}{e\sub{x}{e_x}}{\tau\sub{x}{e_x}}$$

Otherwise, assume
$$(y,\tau)\in \Gamma'$$
So,
$$(y,\sub{x}{e_x}\tau)\in \sub{x}{e_x}\Gamma'$$
Also, $e\sub{x}{e_x}\equiv e \equiv y$.
By which and rule \rtvar, we get
$$\hastype{\Gamma,\sub{x}{e_x}\Gamma'}{e\sub{x}{e_x}}{\tau\sub{x}{e_x}}$$

\item \rtvarbase
Assume 
	$$\hastype{\Gamma, x\colon\tau_x, \Gamma'}{e}{\tau}$$
where $e \equiv y$ and $\tau\equiv\tref{v}{b}{v\eqb{b} y}$.
By inversion 
$$(y,\tref{v}{b}{e'})\in \Gamma, x\colon\tau_x, \Gamma'$$
Assume
$$(y,\tau)\in \Gamma$$
By rule \rtvarbase
$$\hastype{\Gamma,\sub{x}{e_x}\Gamma'}{e}{\tau}$$
Since \iswellformed{}{\Gamma}, $x$ cannot appear in $\tau$
so $\tau\sub{x}{e_x}\equiv\tau$.
Also, $x\neq y$, so $e\sub{x}{e_x}\equiv e$.
So,
$$\hastype{\Gamma,\sub{x}{e_x}\Gamma'}{e\sub{x}{e_x}}{\tau\sub{x}{e_x}}$$

Assume
$$y \equiv x$$
By lemma's assumption 
$$\shastype{\Gamma}{e_x}{\tau_x}$$
and since each expression has at most one unrefined type
$$\hastype{\Gamma,\sub{x}{e_x}\Gamma'}{e_x}{\tref{v}{b}{e''}}$$
By rule \rtexact we get
$$\hastype{\Gamma,\sub{x}{e_x}\Gamma'}{e_x}{\tref{v}{b}{v \eqb{b} e_x}}$$
Since $x = y$, $e\sub{x}{e_x} \equiv e_x$.
Also, $\tref{v}{b}{v=y}\sub{x}{e_x}=\tref{v}{b}{v \eqb{b} e_x}$ 
So,
$$\hastype{\Gamma,\sub{x}{e_x}\Gamma'}{e\sub{x}{e_x}}{\tau\sub{x}{e_x}}$$

Otherwise, assume
$$(y,\tau)\in \Gamma'$$
So,
$$(y,\sub{x}{e_x}\tau)\in \sub{x}{e_x}\Gamma'$$
Also, $e\sub{x}{e_x}\equiv e \equiv y$ and $\tau\sub{x}{e_x} =\tau$.
By which and rule \rtvar, we get
$$\hastype{\Gamma,\sub{x}{e_x}\Gamma'}{e\sub{x}{e_x}}{\tau\sub{x}{e_x}}$$

\item Case \rtconst.
Assume 
	$$\hastype{\Gamma, x\colon\tau_x, \Gamma'}{e}{\tau}$$
where $e \equiv w$ and $\tau\equiv\ty{w}$.
Since $e\sub{x}{e_x} \equiv e$ and $\tau\sub{x}{e_x}\equiv\tau$
$$\hastype{\Gamma,\sub{x}{e_x}\Gamma'}{e\sub{x}{e_x}}{\tau\sub{x}{e_x}}$$

\item Case \rtcase.
Assume 
	$$\hastype{\Gamma, x\colon\tau_x, \Gamma'}{e}{\tau}$$
where $e \equiv \ecase{e}{y}{\dcti}{\overline{y}_i}{e_i}$.
By inversion:
$$
	\hastype{\Gamma, x\colon\tau_x, \Gamma'}{e}{\tref{v}{T}{e_T}}\ (1)\qquad
	\iswellformed{\Gamma, x\colon\tau_x, \Gamma'}{\tau}\ (2)
$$
$$
	\tforalli{T}. \left\lbrace
	\begin{array}{ll}
 	\ty{\dcti} = \overline{x\colon\tau_D}\rightarrow \tref{v}{T}{e'_T} \qquad
 	\theta = \sub{x}{y_i} \cr
	\hastype{\Gamma, x\colon\tau_x, \Gamma', \overline{y_i\colon\theta\ \tau_{\dcti}}, 
				y\colon \tref{v}{T}{e_T \land \theta e'_T}
				}{e_i}{\tau}\ (3i)
	\end{array} \right.
$$
By IH on $(1)$ and $(3i)$ and \ref{proof:sub:wf} on $(2)$
$$
	\hastype{\Gamma, \sub{x}{e_x}\Gamma'}{e\sub{x}{e_x}}{\tref{v}{T}{e_T\sub{x}{e_x}}}\ (4)
	\iswellformed{\Gamma, \sub{x}{e_x}\Gamma'}{\tau\sub{x}{e_x}}\ (5)
$$
$$
	\hastype{\Gamma, \sub{x}{e_x}\Gamma', \overline{y_i\colon\theta\ \tau_{\dcti}\sub{x}{e_x}}, 
				y\colon \tref{v}{T}{e_T \land \theta e'_T}\sub{x}{e_x}
				}{e_i\sub{x}{e_x}}{\tau\sub{x}{e_x}}\ (6i)
$$
Since $x$ does not appear free in $\ty{\dcti}$, or$\ty{\dcti}\sub{x}{e_x} = \ty{\dcti}$, 
we get
$$
	\tforalli{T}. \left\lbrace
	\begin{array}{ll}
 	\ty{\dcti} = \overline{x\colon\tau_{\dcti}}\rightarrow \tref{v}{T}{e'_T} \qquad
 	\theta = \sub{x}{y_i} \cr
	\hastype{\Gamma, \sub{x}{e_x}\Gamma', \overline{y_i\colon\theta\ \tau_{\dcti}}, 
				y\colon \tref{v}{T}{e_T\sub{x}{e_x} \land \theta e'_T}
				}{e_i\sub{x}{e_x}}{\tau\sub{x}{e_x}}
	\end{array} \right.
$$
By which, $(4)$, $(5)$ and rule \rtcase we get
$$\hastype{\Gamma, \sub{x}{e_x}\Gamma'}{e\sub{x}{e_x}}{\tau\sub{x}{e_x}}$$


\item\rtsub
Assume 
	$$\hastype{\Gamma, x\colon\tau_x, \Gamma'}{e}{\tau}$$
By inversion
$$
\hastype{\Gamma, x\colon\tau_x, \Gamma'}{e}{\tau'}\ (1)\qquad
\issubtype{\Gamma, x\colon\tau_x, \Gamma'}{\tau'}{\tau} \ (2)\qquad
\iswellformed{\Gamma, x\colon\tau_x, \Gamma'}{\tau} \ (3)
$$
By IH, \ref{proof:sub:sub} and \ref{proof:sub:wf}
$$
\hastype{\Gamma, \sub{x}{e_x}\Gamma'}{\sub{x}{e_x}e}{\sub{x}{e_x}\tau'}\qquad
\issubtype{\Gamma, \sub{x}{e_x}\Gamma'}{\sub{x}{e_x}\tau'}{\sub{x}{e_x}\tau}
$$
$$
\iswellformed{\Gamma, \sub{x}{e_x}\Gamma'}{\sub{x}{e_x}\tau}
$$
By rule \rtsub
$$\hastype{\Gamma,\sub{x}{e_x}\Gamma'}{e\sub{x}{e_x}}{\tau\sub{x}{e_x}}$$
\item\rtfun
Assume
	$$\hastype{\Gamma, x\colon\tau_x, \Gamma'}{e}{\tau}$$
where $e\equiv\efun{y}{e'}$ and $\tau\equiv\tfun{y}{\tau'_y}{\tau'}$.
By inversion
	$$
	\hastype{\Gamma, x\colon\tau_x, \Gamma', y\colon\tau'_y}{e'}{\tau'}\ (1)\qquad
	\iswellformed{\Gamma, x\colon\tau_x, \Gamma'}{\tau'_y}\ (2)
	$$
By IH and \ref{proof:sub:wf}
	$$
	\hastype{\Gamma,\sub{x}{e_x} \Gamma', y\colon\sub{x}{e_x}\tau'_y}{\sub{x}{e_x}e'}{\sub{x}{e_x}\tau'} \qquad
	\iswellformed{\Gamma, \sub{x}{e_x}\Gamma'}{\sub{x}{e_x}\tau'_y}
	$$
	By rule \rtfun
	$$
	\hastype{\Gamma,\sub{x}{e_x} \Gamma'}{\sub{x}{e_x}e}{\sub{x}{e_x}\tau}
	$$
	
\item\rtapp
Assume
	$$\hastype{\Gamma, x\colon\tau_x, \Gamma'}{e}{\tau}$$
where $e\equiv\eapp{e_1}{e_2}$ and $\tau\equiv\tau'\sub{y}{e_2}$.
By inversion
	$$
	\hastype{\Gamma, x\colon\tau_x, \Gamma'}{e_1}{\tfun{y}{\tau'_y}{\tau'}}\ (1)\qquad
	\hastype{\Gamma, x\colon\tau_x, \Gamma'}{e_2}{{\tau'_y}}\ (2)
	$$
By IH 
	$$
	\hastype{\Gamma,\sub{x}{e_x} \Gamma'}{\sub{x}{e_x}e_1}{\sub{x}{e_x}\tfun{y}{\tau'_y}{\tau'}} \qquad
	\hastype{\Gamma,\sub{x}{e_x} \Gamma'}{\sub{x}{e_x}e_2}{\sub{x}{e_x}{\tau'_y}}
	$$
	By rule \rtapp
	$$
	\hastype{\Gamma,\sub{x}{e_x} \Gamma'}{\sub{x}{e_x}e}{\sub{x}{e_x}\tau}
	$$
\end{itemize}
\item \label{proof:sub:wf}
Assume \iswellformed{\Gamma, x\colon\tau_x, \Gamma'}{\tau}.
We will prove it by induction on the derivation.
\begin{itemize}
\item \rwbase
Assume 
$$\iswellformed{\Gamma, x\colon\tau_x, \Gamma'}{\tau}$$
where $\tau\equiv\tref{v}{b}{e}$.
By inversion
$$\hastypebase{\erase{\Gamma, x\colon\tau_x, \Gamma'},v\colon b}{e}{\tbool}$$
So,
$$\hastypebase{\erase{\Gamma, \sub{x}{e_x}\Gamma'},v\colon b}{e\sub{x}{e_x}}{\tbool}$$
By rule \rwbase
$$\iswellformed{\Gamma, \sub{x}{e_x}\Gamma'}{\tref{v}{b}{e\sub{x}{e_x}}}$$
Or 
$$\iswellformed{\Gamma, \sub{x}{e_x}\Gamma'}{\tau\sub{x}{e_x}}$$
\item \rwfun
Assume
$$\iswellformed{\Gamma, x\colon\tau_x, \Gamma'}{\tau}$$
where $\tau\equiv \tfun{y}{\tau'_y}{\tau'}$.
By inversion, we get
$$
	\iswellformed{\Gamma, x\colon\tau_x, \Gamma'}{\tau_x} \qquad
	\iswellformed{\Gamma, x\colon\tau_x, \Gamma', y \colon \tau'_y}{\tau'}
$$
By IH
$$
	\iswellformed{\Gamma, \sub{x}{e_x} \Gamma'}{\tau_x\sub{x}{e_x}}\qquad
	\iswellformed{\Gamma, \sub{x}{e_x}(\Gamma', y \colon \tau'_y)}{\tau'\sub{x}{e_x}}
$$
Due to $\alpha$-renaming, $x \neq y$, so
$$
	\iswellformed{\Gamma, \sub{x}{e_x} \Gamma'}{\tau'_y\sub{x}{e_x}}\qquad
	\iswellformed{\Gamma, \sub{x}{e_x}\Gamma', y \colon \sub{x}{e_x}\tau'_y}{\tau'\sub{x}{e_x}}
$$
By \rwfun
$$
	\iswellformed{\Gamma, \sub{x}{e_x} \Gamma'}{\tfun{y}{\tau'_y\sub{x}{e_x}}{\tau'\sub{x}{e_x}}}
$$
Or
$$
	\iswellformed{\Gamma, \sub{x}{e_x} \Gamma'}{\tau\sub{x}{e_x}}
$$
\end{itemize}

	then
	\iswellformed{\Gamma, \sub{x}{e_x}\Gamma'}{\sub{x}{e_x}\tau}
\end{enumerate}

	\end{proof}
}

\begin{lemma}\label{lemma:erase} % for \ref{lemma:wellformed}
If \hastype{\Gamma}{e}{\tau} 
then \hastypebase{\erase{\Gamma}}{e}{\erase{\tau}}.
\end{lemma}


\begin{lemma}\label{lemma:wellformed}
If \iswellformed{}{\Gamma} and \hastype{\Gamma}{e}{\tau} then \iswellformed{\Gamma}{\tau}.
\end{lemma}
\showproof{
	\begin{proof}
	Assume
\iswellformed{}{\Gamma} and \hastype{\Gamma}{e}{\tau}.
We will prove the Lemma by induction on the derivation tree.
\begin{itemize}
\item Case \rtexact. Assume 
$$	\hastype{\Gamma}{e}{\tau} $$
where $\tau\equiv\tref{v}{b}{v \eqb{b} e}$.
By inversion
$$	
\hastype{\Gamma}{e}{\tref{v}{b}{e'}}
$$
By Lemma \ref{lemma:erase}
$$	
\hastypebase{\erase{\Gamma}}{e}{b}
$$
By Definition \ref{def:constants}
$$	
\hastypebase{\erase{\Gamma},v\colon b}{v \eqb{b} e}{\tbool}
$$
So, 
$$	
\iswellformed{\Gamma}{\tau}
$$

\item Case \rtvarbase
Assume 
$$	\hastype{\Gamma}{e}{\tau} $$
where $\tau\equiv\tref{v}{b}{v \eqb{b} x}$.
By inversion
$$	
(x,{\tref{v}{b}{e'}}) \in \Gamma
$$

So, 
$$	
(x,b) \in \erase{\Gamma}
$$
Or,
$$	
\hastypebase{\erase{\Gamma}}{x}{b}
$$
By Definition \ref{def:constants}
$$	
\hastypebase{\erase{\Gamma},v\colon b}{v \eqb{b} x}{\tbool}
$$
So, 
$$	
\iswellformed{\Gamma}{\tau}
$$
\item Case \rtvar
Assume $$\hastype{\Gamma}{e}{\tau}$$
By inversion $(x,\tau)\in \Gamma$ and since \iswellformed{}{\Gamma}
$$	
\iswellformed{\Gamma}{\tau}
$$
\item Case \rtconst
Assume $$\hastype{\Gamma}{e}{\tau}$$
where $e\equiv c$ and $\tau\equiv\ty{c}$.
By Definition \ref{def:constants},
\iswellformed{\Gamma}{\ty{c}}

\item Case \rtsub.
Assume 
$$	
	\hastype{\Gamma}{e}{\tau}
$$
By inversion
$$	\iswellformed{\Gamma}{\tau}
$$

\item Case \rtcase.
Assume
$$	
	\hastype{\Gamma}{e}{\tau}
$$
where $e \equiv \ecase{e'}{x}{\dcti}{\overline{y_i}}{e_i}$.
By inversion
$$	\iswellformed{\Gamma}{\tau}
$$
\item Case \rtfun.
Assume
$$
	\hastype{\Gamma}{e}{\tau}
$$
where $\tau\equiv\tfun{x}{\tau_x}{\tau'}$ 
and $e\equiv\efun{x}{e_x}$.

By inversion
$$
	\hastype{\Gamma, x\colon\tau_x}{e}{\tau'} \ (1) \qquad
	\iswellformed{\Gamma}{\tau_x}\ (2)
$$
By IH on $(1)$
$$
	\iswellformed{\Gamma, x\colon\tau_x}{\tau'}
$$
By which, $(2)$ and rule \rwfun
$$
	\iswellformed{\Gamma}{\tau}
$$

\item Case \rtapp. Assume
$$
	\hastype{\Gamma}{e}{\tau}
$$
where $\tau\equiv\tau'\sub{x}{e_2}$ 
and $e\equiv\eapp{e_1}{e_2}$.
By inversion
$$
	\hastype{\Gamma}{e_1}{(\tfun{x}{\tau_{x}}{\tau'})}\ (1) \qquad 
	\hastype{\Gamma}{e_2}{\tau_{x}}\ (2)
$$
By IH on $(1)$
$$
	\iswellformed{\Gamma}{(\tfun{x}{\tau_{x}}{\tau'})}
$$
By inversion on \rtapp
$$
	\iswellformed{\Gamma, x\colon\tau_x}{\tau'}
$$
By which, $(2)$ and Lemma \ref{lemma:substitution}
$$
	\iswellformed{\Gamma}{\tau}
$$
\end{itemize}
	\end{proof}
}

\subsection{Soundness}
\begin{lemma}\label{lemma:eval}
If \eval{e}{e'} then \issubtype{\Gamma}{\tau\sub{x}{e'}}{\tau\sub{x}{e}}.
\end{lemma}
\showproof{
	\begin{proof}
	Assume \eval{e}{e'}.
We will prove that 
\issubtype{\Gamma}{\tau\sub{x}{e'}}{\tau\sub{x}{e}}
and
\issubtype{\Gamma}{\tau\sub{x}{e}}{\tau\sub{x}{e'}}
by structural induction on $\tau$.

\begin{itemize}
\item $\tau\equiv \tref{v}{b}{e_r}$
By Definition \ref{def:valid}, 
$$\forall \theta. \iswellformed{\Gamma}{\theta} \land 
	\generalconditionImpl{\theta\ e_r\sub{x}{e'}}{\theta\ e_r\sub{x}{e}}
$$
$$
\forall \theta. \iswellformed{\Gamma}{\theta} \land 
	\generalconditionImpl{\theta\ e_r\sub{x}{e}}{\theta\ e_r\sub{x}{e'}}$$

So, 
$$\issubtype{\Gamma}{\tau\sub{x}{e'}}{\tau\sub{x}{e}}
\qquad
\issubtype{\Gamma}{\tau\sub{x}{e}}{\tau\sub{x}{e'}}
$$	
\item $\tau\equiv \tfun{y}{\tau'_y}{\tau'}$
By IH we have
$$
\issubtype{\Gamma}{\tau'_y\sub{x}{e'}}{\tau'_y\sub{x}{e}}\ (1)
\qquad
\issubtype{\Gamma}{\tau'_y\sub{x}{e}}{\tau'_y\sub{x}{e'}}\ (2)
$$

$$
\issubtype{\Gamma, y\colon\tau'_y\sub{x}{e}}{\tau'\sub{x}{e'}}{\tau'\sub{x}{e}}\ (3)
\qquad
\issubtype{\Gamma, y\colon\tau'_y\sub{x}{e'}}{\tau'\sub{x}{e}}{\tau'\sub{x}{e'}}\ (4)
$$

By $(1), (4)$ and rule \rtsub we have
$$\issubtype{\Gamma}{\tau\sub{x}{e}}{\tau\sub{x}{e'}}$$

By $(2), (3)$ and rule \rtsub we have
$$\issubtype{\Gamma}{\tau\sub{x}{e'}}{\tau\sub{x}{e}}$$
\end{itemize}

	\end{proof}
}

\begin{lemma}[Preservation]\label{lemma:preservation}
If \hastype{\emptyset}{e}{\tau} and \eval{e}{e'} then \hastype{\emptyset}{e'}{\tau}.
\end{lemma}
\showproof{
	\begin{proof}
	Assume \hastype{\emptyset}{e}{\tau} and \eval{e}{e'}. 
We will prove the lemma by induction on the derivation tree. 
\begin{itemize}
\item Case \rtexact. Assume $$\hastype{\emptyset}{e}{\tau}\ (1)$$
where $\tau \equiv\tref{v}{b}{v \eqb{b} e}$.

By inversion
$$\hastype{\emptyset}{e}{\tref{v}{b}{e_v}}$$
By IH
$$\hastype{\emptyset}{e'}{\tref{v}{b}{e_v}}$$
By rule \rtexact
$$\hastype{\emptyset}{e'}{\tref{v}{b}{v\eqb{b} e'}} \ (2)$$
By Lemma \ref{lemma:eval}
$$\issubtype{\emptyset}{\tref{v}{b}{v\eqb{b} e'}}{\tref{v}{b}{v\eqb{b} e}} \ (3)$$
By Lemma \ref{lemma:wellformed} on $(1)$ (since \iswellformed{}{\emptyset})
$$\iswellformed{\emptyset}{\tref{v}{b}{v\eqb{b} e}} \ (4)$$
By $(2), (3), (4)$ and rule \rtsub:
$$\hastype{\emptyset}{e'}{\tref{v}{b}{v\eqb{b} e}}$$

\item Cases \rtvarbase, \rtvar, \rtconst and \rtfun trivially hold
       as there is no $e'$ such that \eval{e}{e'}.

\item Case \rtsub. Assume
$$	\hastype{\emptyset}{e}{\tau}$$
By inversion
$$	\hastype{\emptyset}{e}{\tau'} \ (1) \qquad
	\issubtype{\emptyset}{\tau'}{\tau}\ (2) \qquad
	\iswellformed{\emptyset}{\tau}\ (3)
$$

By IH on $(1)$
$$	\hastype{\emptyset}{e'}{\tau'} $$
By which, $(2), (3)$ and \rtsub
$$	\hastype{\emptyset}{e'}{\tau}$$


\item Case \rtapp. Assume
$$	\hastype{\emptyset}{e}{\tau}\ (1)$$
where $e \equiv \eapp{e_1}{e_2}$, and
	  $\tau\equiv\tau'\sub{x}{e_2}$

By inversion
$$	
	\hastype{\emptyset}{e_1}{(\tfun{x}{\tau_{x}}{\tau'})}\ (2) \qquad
	\hastype{\emptyset}{e_2}{\tau_{x}}\ (3)
$$

We split cases on the structure of $e$.
\begin{itemize}
\item $e\equiv \eapp{c}{v_2}$.
Then, $e'\equiv\interp{c}(v_2)$.
By Definition \ref{def:constants},
$$\hastype{\emptyset}{e'}{\tau}$$

\item $e\equiv \eapp{c}{e_2}$ where $e_2$ is not a value, 
Then, by (3) and Lemma \ref{lemma:progress},
\eval{e_2}{e_2'}, and $e' \equiv \eapp{e_1}{e_2'}$
%
By IH on $(2)$
$$	\hastype{\emptyset}{e_2'}{\tau_{x}}$$
By which, $(1)$ and rule \rtapp we get
$$\hastype{\emptyset}{e'}{\tau'\sub{x}{e_2'}}\ (4)$$
By Lemma \ref{lemma:eval}
$$
	\issubtype{\emptyset}{\tau'\sub{x}{e_2'}}{\tau'\sub{x}{e_2}}\ (5)
$$
By $(1)$ and Lemma \ref{lemma:wellformed}, since \iswellformed{}{\emptyset}
$$
	\iswellformed{\emptyset}{\tau'\sub{x}{e_2}}\ (6)
$$
By $(4), (5), (6)$ and rule \rtsub
$$	\hastype{\emptyset}{e'}{\tau}$$

\item $e \equiv \eapp{\efun{x}{e_x}}{e_2}$.
Then, $e' \equiv e_x\sub{x}{e_2}$.

By inversion on $(2)$
$$
	\hastype{x\colon\tau_x}{e_x}{\tau'}
$$
By which, $(3)$ and Lemma \ref{lemma:substitution} (since \iswellformed{}{x\colon\tau_x})
$$\hastype{\emptyset}{e'}{\tau'}$$

\item $e \equiv \eapp{e_1}{e_2}$, where $e_1$ is not a value.
Then, by $(2)$ and Lemma \ref{lemma:progress}, \eval{e_1}{e_1'} and 
$e'\equiv\eapp{e_1'}{e_2}$
By IH on $(2)$
$$	\hastype{\emptyset}{e_1'}{(\tfun{x}{\tau_{x}}{\tau'})}
$$
By which, $(3)$ and rule \rtapp we get
$$	\hastype{\emptyset}{e'}{\tau}$$
\end{itemize}
\end{itemize}
	\end{proof}
}
\begin{lemma}[Progress]\label{lemma:progress}
If \hastype{\emptyset}{e}{\tau} and $e \neq v$ then there exists an $e'$ such that \eval{e}{e'}.
\end{lemma}
\showproof{
	\begin{proof}
	Assume \hastype{\emptyset}{e}{\tau}.
We will prove the Lemma by induction on the derivation tree.
\begin{itemize}
\item Case \rtexact. Assume
$$	\hastype{\emptyset}{e}{\tref{v}{b}{v = e}}
$$
where $\tau\equiv\tref{v}{b}{v=e}$.
By inversion
$$	\hastype{\emptyset}{e}{\tref{v}{b}{e'}}$$
By IH 
either $e \equiv v$ or there exists an $e'$ such that \eval{e}{e'}.

\item Cases \rtvarbase, \rtvar cannot occur, as $\Gamma = \emptyset$
\item Cases \rtconst and \rtfun are trivial, 
		as $e$ is a value
\item Case \rtsub. Assume 
$$\hastype{\emptyset}{e}{\tau}$$
By inversion
$$	\hastype{\emptyset}{e}{\tau'}$$
By IH 
either $e \equiv v$ or there exists an $e'$ such that \eval{e}{e'}.
\item Case \rtapp. Assume 
$$\hastype{\emptyset}{e}{\tau}\ (1)$$
where $e\equiv\eapp{e_1}{e_2}$ and $\tau\equiv\tau'\sub{x}{e_2}$
By inversion
$$
	\hastype{\emptyset}{e_1}{(\tfun{x}{\tau_{x}}{\tau})}\ (2)\qquad
	\hastype{\emptyset}{e_2}{\tau_{x}}\ (3)
$$

We split cases on the structure of $e$.
\begin{itemize}
\item $e\equiv \eapp{c}{v_2}$.
Then, $e'\equiv\interp{c}(v_2)$.

\item $e\equiv \eapp{c}{e_2}$ where $e_2$ is not a value, 
By IH on $(3)$ \eval{e_2}{e_2'} and  $e' \equiv \eapp{e_1}{e_2'}$

\item $e \equiv \eapp{\efun{x}{e_x}}{e_2}$.
Then, $e' \equiv e_x\sub{x}{e_2}$.

\item $e \equiv \eapp{e_1}{e_2}$, where $e_1$ is not a value.
Then, by IH on $(2)$ \eval{e_1}{e_1'} and 
$e'\equiv\eapp{e_1'}{e_2}$.
\end{itemize}
\end{itemize}
	\end{proof}
}

\renewcommand{\isDecidable}{true} % true or false

\section{Decidable System}
\subsection*{Typing}
\hfill\mbox{\hastype{\Gamma}{e}{\tau}}
%\rtvar
%\rtvarbase
%\rtconst
%\rtsub
%\rtexact
%\rtfun
%\rtapp
%\rtcase
%\rtlet
$$
\inference{
	\hastype{\Gamma}{e}{\tref{v}{b}{e'}}
}{
	\hastype{\Gamma}{e}{\tref{v}{b}{v \eqb{b} e}}
}[\rtexact]
$$
$$
\inference{
	(x,\tref{v}{b}{e}) \in \Gamma
}{
	\hastype{\Gamma}{x}{\tref{v}{b}{v \eqb{b} x}}
}[\rtvarbase]
\qquad
\inference{
	(x,\tau) \in \Gamma &&
	\tau \neq (x,\tref{v}{b}{e})
}{
	\hastype{\Gamma}{x}{\tau}
}[\rtvar]
$$
$$
\inference{
}{
	\hastype{\Gamma}{w}{\ty{w}}
}[\rtconst]
\qquad
\inference{
	\hastype{\Gamma}{e}{\tau'} &&
	\issubtype{\Gamma}{\tau'}{\tau} &&
	\iswellformed{\Gamma}{\tau} &&
}{
	\hastype{\Gamma}{e}{\tau}
}[\rtsub]
$$
$$
\inference{
	\hastype{\Gamma, x\colon\tau_x}{e}{\tau} &&
	\iswellformed{\Gamma}{\tau_x}
}{
	\hastype{\Gamma}{\efun{x}{e}}{(\tfun{x}{\tau_x}{\tau})}
}[\rtfun]
\qquad
\inference{
	\hastype{\Gamma}{e_1}{(\tfunref{x}{\tau_{x}}{\tau}{v}{e_v})} &&
	\hastype{\Gamma}{e_2}{\tau_{x}}
}{
	\hastype{\Gamma}{\eapp{e_1}{e_2}}{\tau\sub{x}{e_2}}
}[\rtapp]
$$
$$
\inference{
	\hastype{\Gamma}{e_x}{\tau_{x}} &&
	\hastype{\Gamma,x\colon\tau_x}{e_2}{\tau} &&
	\iswellformed{\Gamma}{\tau}
}{
	\hastype{\Gamma}{\elet{x}{e_x}{e}}{\tau}
}[\rtlet]
$$
$$
\inference{
	\hastype{\Gamma}{e}{\tref{v}{T}{e_T}} && \iswellformed{\Gamma}{\tau}\\
	\tforalli{T}. \left\lbrace
	\begin{array}{ll}
 	\ty{\dcti} = \overline{x\colon\tau_{\dcti}}\rightarrow \tref{v}{T}{e'_T} \qquad
 	\theta = \sub{x}{y_i} \cr
	\hastype{\Gamma, \overline{y_i\colon\theta\ \tau_{\dcti}}, 
				x\colon \tref{v}{T}{e_T \land \theta e'_T}
				}{e_i}{\tau}
	\end{array} \right.
}{
	\hastype{\Gamma}{\ecase{e}{x}{\dcti}{\overline{y}_i}{e_i}}{\tau}
}[\rtcase]
$$

\hfill\mbox{\iswellformed{\Gamma}{\tau}}
$$
\inference{
	\hastypebase{\erase{\Gamma}, v\colon b}{e}{\tbool}
}{
	\iswellformed{\Gamma}{\tref{v}{b}{e}}
}[\rwbase]
\qquad
\inference{
%%TFUNREF	\hastypebase{\erase{\Gamma}, v\colon b}{e}{\tbool} &&
	\iswellformed{\Gamma}{\tau_x} &&
	\iswellformed{\Gamma, x \colon \tau_x}{\tau}
}{
	\iswellformed{\Gamma}{\tfunref{x}{\tau_x}{\tau}{v}{e}}
}[\rwfun]
$$

\hfill\mbox{\issubtype{\Gamma}{\tau}{\tau}}
$$
\inference{
	\issubref{\Gamma, v:b}{e_1}{e_2}
}{
	\issubtype{\Gamma}{\tref{v}{b}{e_1}}{\tref{v}{b}{e_2}}
}[\rsubbase]
\qquad
\inference{
%% FUNREF	\issubref{\Gamma, v:\tfun{x}{\tau_x}{\tau}}{e_1}{e_2} \\
	\issubtype{\Gamma}{\tau'_x}{\tau_x} &&
	\issubtype{\Gamma, x \colon \tau'_x}{\tau}{\tau'}
}{
	\issubtype{\Gamma}{\tfunref{x}{\tau_x}{\tau}{v}{e_1}}{\tfunref{x}{\tau'_x}{\tau'}{v}{e_2}}
}[\rsubfun]
$$

\hfill\mbox{\issubref{\Gamma}{e}{e}}
$$
\inference{
	\forall \theta. \iswellformed{\Gamma}{\theta} \land
	\generalconditionImpl{\theta\ e_1}{\theta\ e_2}
}{
	\issubref{\Gamma}{e_1}{e_2}
}[\rimpl]
$$

\hfill\mbox{\iswellformed{}{\Gamma}}
$$
\inference{
	\iswellformed{}{\Gamma} &&
	\iswellformed{\Gamma}{\tau}
}{
	\iswellformed{}{x\colon\tau,\Gamma}
}
\qquad
\inference{}{\iswellformed{}{\emptyset}}
$$

\hfill\mbox{\iswellformed{\Gamma}{\theta}}
$$
\inference{
	\forall x \in \text{Dom}(\Gamma). 
	\theta(x) \in \interp{\theta \ \Gamma(x)}
}{
	\iswellformed{\Gamma}{\theta}
}
$$

\subsection{Interpretations}
\begin{align*}
\valid{e} \Leftrightarrow \evals{e}{v}\Rightarrow\evals{e}{\etrue}
\end{align*}

\subsection{Constants}

Now, we should prove that each constant we define respects 
the definition of constants.

\paragraph{Note} that 
if $\valid{e} \Leftrightarrow \etrue$
then each $\hastype{\emptyset}{v}{\tref{v}{b}{\efalse}}$
so @error :: {v: b | false } -> T@ should be defined for each value.

With the above definition for \validi{\star} we can show 
that there is no value $v$ such that \hastype{\Gamma}{v}{\tref{v}{b}{\efalse}}

\newcommand\prop{\ensuremath{\mathtt{prop}}}
\begin{itemize}
\item 
\begin{verbatim}
prop :: Bool -> bool
prop(True) = true
prop(False) = false
\end{verbatim}

\item 
\begin{verbatim}
<=> :: bool -> bool -> bool
true  <=> true  = true
false <=> false = true
true  <=> false = false
false <=> true  = false
\end{verbatim}

\item $\texttt{Bool}\in T$, $i_\texttt{Bool} = 2$
\item $\texttt{True}  \equiv D^{\texttt{Bool}}_1, \texttt{False} \equiv D^{\texttt{Bool}}_2$
\item $\ty{\texttt{True}} = \tref{v}{\texttt{Bool}}{\prop\ v \Leftrightarrow \etrue}$, and
$\ty{\texttt{False}} = \tref{v}{\texttt{Bool}}{\prop\ v \Leftrightarrow \efalse}$

\item 
\begin{verbatim}
error :: {v:b | false} -> T
error _ = crash
\end{verbatim}

Valid as $\evals{e}{v} \Rightarrow e \notin \interp{\tref{v}{b}{\efalse}}$

\item 
\begin{align*}
\efix{\tau} & : (\tau\rightarrow\tau) \rightarrow \tau\\
\efix{\tau} f &=\eapp{f}{(\eapp{\efix{\tau}}{f})}
\end{align*}
{% \showproof{
\input{proofs/fix}
}
\end{itemize}

\subsection{Logic}
\begin{align*}
\mathcal{C}_0 &= \{ \etrue, \efalse, 0, 1, \dots \} \\
\mathcal{C}_i &= \{ f, +, -, = , >, \lnot, \land, \dots \}
\end{align*}

\hfill\mbox{\ispure{\Gamma}{e}}
$$
\inference{
	(x,b) \in \Gamma
}{
	\ispure{\Gamma}{x}
}
\qquad
\inference{
	c_n \in \mathcal{C}_n &&
	\forall i. 1 \leq i \leq n \Rightarrow \ispure{\Gamma}{e_i}
}{
	\ispure{\Gamma}{c_n\ e_1\ \dots e_n}
}
$$

%%\begin{align*}
%%\interp{x}&=x &&&
%%\interp{\efun{x}{e}}&=f\\
%%\interp{c}&=c &&&
%%\interp{\eapp{e_1}{e_2}}&=\interp{e_1}(\interp{e_2})\\
%%\end{align*}
%%
%%\begin{claim}
%%$$
%%\left\lbrace	
%%	\bigwedge_{(x,\tref{b}{v}{e})\in\Gamma}(\fin{x}\Rightarrow \interp{e\sub{v}{x}})
%%	\Rightarrow \interp{e_1}
%%	\Rightarrow \interp{e_2}
%%\right\rbrace
%%\Rightarrow
%%\left\lbrace	
%%\issubref{\Gamma}{e_1}{e_2}
%%\right\rbrace
%%$$
%%\end{claim}
\end{document}
